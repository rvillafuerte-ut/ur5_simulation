\customchapter{RESUMEN}

El presente trabajo de investigación describe el diseño, implementación e integración de un sistema robótico teleoperado con retroalimentación háptica, orientado a la ejecución de suturas superficiales en un entorno controlado. El sistema está conformado por dos brazos robóticos UR5, dos dispositivos hápticos Geomagic Touch y adaptaciones mecánicas para el uso de pinzas laparoscópicas genéricas. La arquitectura maestro-esclavo desarrollada permite que un operador controle de forma remota el efector final del robot, percibiendo además la fuerza de interacción mediante estímulos hápticos.

Se plantearon e implementaron diferentes estrategias de control para el seguimiento de trayectorias, incluyendo control por optimización y control por modo deslizante (SMC), evaluando su desempeño en términos de precisión y estabilidad. Asimismo, se diseñaron y fabricaron adaptadores personalizados tanto para los grippers Robotiq como para los dispositivos hápticos, permitiendo una interacción adecuada con los instrumentos quirúrgicos.

El sistema fue validado mediante pruebas experimentales sobre kits de entrenamiento médico, demostrando su capacidad para ejecutar tareas de sutura superficial de manera precisa, estable y repetible. Los resultados obtenidos evidencian el potencial del sistema como herramienta de asistencia o entrenamiento, y sientan las bases para futuras investigaciones orientadas a su uso en contextos clínicos reales. \\


\noindent \textbf{Palabras clave:}\
\noindent Teleoperación; Control háptico; Sutura superficial; Robot UR5; Control por optimización; Control por modo deslizante.