\customchapter{CONCLUSIONES} 
El Banco de pruebas implementado para realizar las pruebas de sutura integran un sistema de Teleoperación para el robot UR5 que permite comandar al robot con Posiciones y Orientaciones distintas a lo largo de una trayectoria creada desde la herramienta Háptica Geomagic Touch. Para lograr un correcto funcionamiento de este sistema se realiza el diseño de 3 arquitecturas de control basadas en un control cinemático y dos controles Dinámicos. El control cinemático usa un optimizador Jerárquico y los controles dinámicos se modelan con base en el Controlador por Modos Deslizantes y el control por Impedancia. 

Durante las pruebas con el control cinemático, se observó un error máximo de ±0.05 metros en los tres ejes cartesianos durante trayectorias con curvas cerradas o movimientos bruscos. A pesar de estos errores, el tiempo de procesamiento y cálculo de las posiciones articulares fue notablemente bajo, menor a 8 ms en trayectorias suaves. Esto permitió una frecuencia de envío de datos al robot de 0.1 kHz, lo cual es esencial para un comportamiento preciso en tareas que demandan alta exactitud.

La implementación del controlador basado en optimización Jerárquica (HQP) demostró una mejora significativa en la precisión. Los errores en los tres ejes cartesianos del espacio de trabajo se redujeron a un máximo de 8 mm, manteniéndose dentro del rango de precisión establecido para un seguimiento de trayectoria adecuado.

En contraste, las arquitecturas de control dinámicas, específicamente el control dinámico SMC y el control por impedancia, mostraron errores máximos de 18 mm y 40 mm, respectivamente, en el seguimiento de trayectorias. Si bien estos valores se encuentran dentro del rango aceptable, son considerablemente superiores a los obtenidos con el control cinemático con optimizador, lo que sugiere que para tareas que priorizan la precisión en el seguimiento, las arquitecturas dinámicas podrían requerir una sintonización más fina o ser menos adecuadas.

Finalmente, las pruebas de suturas realizadas sobre un kit de cirugía laparoscópica, integrando todos los elementos del banco de prueba, confirmaron la viabilidad del sistema en un entorno controlado. Se logró un tiempo medio de finalización de 5 minutos con errores máximos de 8 mm en el seguimiento de trayectoria. Es importante destacar la aparición de 3 episodios de Jacobianos indeterminados durante movimientos bruscos, un aspecto a considerar para la robustez del sistema en situaciones dinámicas.
