\chapter{REVISIÓN CRÍTICA DE LA LITERATURA}

% Se realiza un análisis crítico de la literatura dirigida a robots quirúrgicos con el objetivo de identificar mecanismos, consideraciones técnicas y aplicaciones similares al procedimiento de suturas quirúrgicas. Asimismo, se investigan los distintos métodos de control utilizados para este tipo de sistemas hápticos, ya que, debido a la no linealidad del sistema y los requerimientos de estabilidad en su funcionalidad(movimiento, torque aplicado, seguimiento de trayectoria), no todos los métodos de control muestran un desempeño adecuado en la implementación.
% \cite{GONZALEZ-CELY2018}


%Sugerencia de cambio por la segunda revisión

En este capítulo se realiza una evaluación técnica de los antecedentes sobre robots quirúrgicos, con el objetivo de identificar mecanismos, configuraciones técnicas y aplicaciones relevantes al procedimiento de suturas. Este análisis se fundamenta en la revisión detallada de sistemas representativos que aporten bases concretas para el diseño metodológico y la interpretación de resultados en este proyecto.

Asimismo, se analizan los métodos de control aplicados a sistemas hápticos, evaluando su desempeño frente a las características no lineales del sistema y los requerimientos de estabilidad en torque, movimiento y seguimiento de trayectorias. Este enfoque permite identificar limitaciones técnicas y estrategias de control \cite{GONZALEZ-CELY2018} con potencial de adaptación al contexto de esta investigación.


\section{Antecedentes} %min 10 años de antiguedad,  solo hitos 

% Corrección avance2: le puse la ref que faltaba 

% La búsqueda de un sistema que disminuya los riesgos en una intervención quirúrgica ha permitido el desarrollo de diversos sistemas, técnicas y tecnologías a lo largo de la historia médica. Es bajo esta búsqueda que la robótica aparece como solución y se convierte en un tópico que ha resonado en las últimas décadas por las posibilidades que su desarrollo ofrece en el campo de las intervenciones no invasivas y de precisión. 
% La investigación en robótica para trabajos en laparoscopia es uno de los pilares dentro de los avances en cirugías no invasivas, tal y como se menciona en \cite{Historia_de_los_robots_cirujanos} donde menciona hasta un incremento de 10 veces dentro del campo de las cirugías laparoscópicas. Investigaciones como las realizadas con el robot DaVinci \cite{ma_da_2024} han mostrado resultados alentadores y han impulsado al desarrollo de nuevos modelos de robots y nuevas técnicas para la aplicación de sistemas de control, como el control por impedancia. Por ejemplo, un último hito fue la realización exitosa de una cirugía para la extracción de un tumor benigno en China \cite{ma_da_2024}, donde el uso del robot DaVinci fue crucial para evitar la sobremanipulación del área a operar. 

%Asimismo, a la par del desarrollo de sistemas robóticos en cirugía se puede encontrar el desarrollo de la manipulación de los mismos, la teleoperación. La robótica en cirugía a lo largo de las distintas aplicaciones y modelos desarrollados busca perfeccionar la 

% Propuesta de corrección:

La búsqueda de sistemas que reduzcan los riesgos durante intervenciones quirúrgicas ha impulsado el desarrollo de diversas técnicas y tecnologías a lo largo de la historia médica. En este contexto, la robótica ha emergido como una solución clave, destacándose en las últimas décadas por su potencial para optimizar procedimientos quirúrgicos no invasivos y de alta precisión

Según \cite{Historia_de_los_robots_cirujanos}, la integración de robots en este campo ha llevado a un aumento de hasta 10 veces en la precisión de las intervenciones laparoscópicas. Un caso destacado es el robot Da Vinci, cuya investigación y desarrollo han dado lugar a técnicas innovadoras como el control por impedancia, diseñado para ajustar dinámicamente la fuerza aplicada por los manipuladores robóticos en respuesta a las interacciones con el tejido. De esta forma, se evitaría traspasar o dañar los tejidos por la falta de este método de control. 

% FAlta corregir la parte que pregunta si es una investigación, asúmo que con decir un poco más de la fuente basta pero no estoy seguro uu, borrar "podemos"





\section{Estado del Arte}

\subsection{Sistemas robóticos teleoperados}
La teleoperación permite la manipulación del movimiento de un robot sin la necesidad de centrarse en el control interno del mismo \cite{you_assisted_2012}, enfatizando el trabajo del control a un nivel más alto y permitiendo realizar tareas más complejas a través del sistema Maestro-Esclavo.
Dentro de las investigaciones de teleoperación para robots podemos encontrar el trabajo en la obtención del movimiento del operario, siendo la correcta detección de esta un punto crucial al momento de traducir un movimiento natural en ángulos de rotación de un brazo robótico. Una investigación realizada para el simposio internacional de Electrónica en 2019 testeó el control de un brazo de 4 grados de libertad  mediante la herramienta Kinetic desarrollado por Microsoft \cite{syakir_teleoperation_2019} para reemplazar el sistema Maestro convencional. 



% cirujanos podemos encontrar los modelos comerciales, en los que encontramos al robot DaVinci, Zeus, Hugo RAS \cite{prata_state_2023} 

\subsection{Sistemas robóticos aplicados a cirugías}
La inclusión de robots ha iniciado una nueva era en campo médico, especialmente orientado a cirugías, ya que estos permiten un control de un manipulador que permite que el médico cirujano no esté necesariamente presente en el lugar de la operación. Asimismo, estos se han adaptado para reducir su tamaño, de tal forma, que no es necesario realizar cortes profundos en para una operación, sino que gracias a esta modificación el brazo robótico cuenta con una extensión suficientemente larga para llegar al lugar deseado para la operación. Asimismo, en cirugías laparoscópicas, por ejemplo, usualmente cuenta con cámaras que le permiten al médico obtener información sobre el entorno sobre el que va a operar. Robots como el Davinci fueron específicamente diseñados para su uso en entornos de operaciones y, a pesar de su elevado costo. Muchos hospitales tomaron la decisión de incluirlos durante sus operaciones, ya que además de minimizar el tamaño de la incisión, ayudan a mejorar la precisión de los movimientos del cirujano \cite{review_of_haptic_feedback}. 





\subsection{Reconocimiento y visión computacional en cirugías}
Desde el descubrimiento de las imágenes de rayos X para el escaneo del cuerpo humano desde 1895 que se usaron para un diagnóstico médico se comenzó a requerir un análisis de estas imágenes y con el avance de la tecnología, surgieron numerosos exámenes que resultaban en la imagen sobre alguna parte del cuerpo humano. El análisis de estas imágenes era de suma importancia, ya que ayudarían a reconocer la presencia de alguna herida u objeto maligno en el cuerpo humano. Sin embargo, ello también dio paso a numerosas investigaciones sobre el análisis de estas imágenes con un enfoque computacional. En este ámbito machine learning no resulta de mucha ayuda, ya que métodos basados en regresiones lineales o árboles de decisión no resultan suficientes. Por ello, a las imágenes médicas se aplica mejor algoritmos con redes convulucionales que permiten la extracción de las características de las imágenes que usualmente no tiene gran calidad. De esta forma, inclusive, se puede buscar detectar la aparición de enfermedades antes de mostrar los signos de esta sobre el cuerpo humano. Estos algoritmos necesitan de una gran cantidad de bases de datos para aprender de estas y realizar las detecciones con mayor precisión, ya que con él cada imagen que analiza no deja de aprender sobre sus errores y continúa mejorando. Por ello, es posible esperar que su influencia en el ámbito médico llegue a causar un gran impacto.


\subsection{Inclusión de la tecnología Háptica}


La tecnología háptica se define como la disciplina que combina la percepción sensorial del tacto con el control de la respuesta a este estímulo, permitiendo diversas aplicaciones tecnológicas \cite{Informacion_de_aplicaciones_tecnologia_haptica}. Esta tecnología busca transmitir las sensaciones físicas al usuario mediante la información recuperada de un modelo vitural o sensores, recreando experiencias que emulan la percepción directa como si el usuario interactuara presencialmente con el entorno. Su aplicación se ha destacado principalmente en la simulación de entornos virtuales para el entrenamiento del el uso de algún dispositivo como en la conducción de vehículos \cite{Ejemplo_haptic_manejo_carro}, manejo de aeronaves \cite{Ejemplo_haptic_piloto_avion} y en el área de capacitación en la industria espacial \cite{Ejemplo_haptic_entrenamiento_espacial}.


En el ámbito médico, las operaciones requieren la plena atención del médico, especialmente en los aspectos visual y háptico, lo que permite un buen desempeño para realizar correctamente la actividad. Asimismo, la robótica se ha introducido en este sector, como es el caso del robot RIO System, que cuenta con retroalimentación háptica y se utiliza en cirugías ortopédicas, o el robot ALF-X, orientado a procedimientos RMIS (cirugía mínimamente invasiva asistida por robot) \cite{robotica_haptics_retos_beneficios}. De este modo, aunque los robots manipuladores para cirugía comenzaron siendo únicamente teleoperados, muchos están incorporando la retroalimentación háptica debido al desempeño que esta proporciona. Al tratar con tejido blando de los órganos, la fuerza que se aplica con el robot debe ser precisa, ya que es posible dañar estos tejidos si no se controla adecuadamente. Solo con la retroalimentación visual podrían ocurrir errores al estimar la profundidad. Esto se refleja en la referencia, donde se menciona que, al incluir la retroalimentación háptica, los daños generados en un simulador de operaciones se redujeron en un 55% \cite{robotica_haptics_retos_beneficios}.

