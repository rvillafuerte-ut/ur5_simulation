%% --------------------------------------------------------------------
%% Template UTEC Tesis
%% --------------------------------------------------------------------
% Este es el archivo que se debe compilar.
% 
% Este template ha sido modificado y actualizado por Eduardo Castro y Roosevelt Ubaldo en base a lo trabajado por Víctor Murray, Oscar Ramos y Juan Carlos Barbaran.
%
% Última actualización: Mayo, 2022

\documentclass[a4paper,12pt,oneside]{tesisutec}

\selectlanguage{spanish}
\usepackage{tikz}
\usetikzlibrary{shapes, arrows.meta, positioning}

%% Paquetes
\usepackage[utf8]{inputenc}
\usepackage[square, numbers, comma, sort&compress]{natbib}
% Libreria de idioma
\usepackage[spanish]{babel}
% Libreria para posicionamiento
\usepackage{float}
% Librerias para insertar codigos
\usepackage[spanish,onelanguage,ruled,vlined]{algorithm2e}
\usepackage{verbatim} 
% Librería para hipervínculos
\usepackage{hyperref}
 % Librería necesaria para arreglar el orden de referencias en overleaf.com
 \usepackage{graphicx}
\usepackage{caption}
\usepackage{notoccite}
\usepackage{subcaption}
\usepackage{optidef}

% Incluir acá los paquetes adicionales que deseas. 
% Ubicación de los imágenes.
\graphicspath{images/}

\makeatletter
% Reinsert missing \algbackskip
\def\algbackskip{\hskip-\ALG@thistlm}
\makeatother

\hypersetup{urlcolor=blue, colorlinks=true}

\begin{document}

\frontmatter

\department{INGENIERÍA MECATRÓNICA}
\degree{Ingeniero }
\major{Ingeniería Mecatrónica}

\title{DESARROLLO DE UN MÓDULO DE CONTROL HÁPTICO TELEOPERADO EN UN SISTEMA ROBÓTICO EN SUTURAS DE ÁREAS SUPERFICIALES}

\author{David Valdez Pampañaupa \orcid{0009-0005-1868-3422} \\ Rodrigo Villafuerte\orcid{0000-0004-0000-0000}} % Es obligatorio agregar ORCID del alumno
\supervisor{Ruth Canahuire Cabello \orcid{0000-0000-0000-0000}} % Es obligatorio agregar ORCID del asesor

\date{2025}

\maketitle

\setstretch{1.5}


\input{encabezados/dedicatoria}
\input{encabezados/agradecimientos}

\tableofcontents

\newpage
valores agregados
SCRMAPER => Valores agregados

\listoftables

\newpage

\listoffigures

\addtocontents{toc}{\vspace{1.5em}}
°
%% ============================================================================
\mainmatter
\pagestyle{fancy}

\customchapter{RESUMEN}

El presente trabajo de investigación describe el diseño, implementación e integración de un sistema robótico teleoperado con retroalimentación háptica, orientado a la ejecución de suturas superficiales en un entorno controlado. El sistema está conformado por dos brazos robóticos UR5, dos dispositivos hápticos Geomagic Touch y adaptaciones mecánicas para el uso de pinzas laparoscópicas genéricas. La arquitectura maestro-esclavo desarrollada permite que un operador controle de forma remota el efector final del robot, percibiendo además la fuerza de interacción mediante estímulos hápticos.

Se plantearon e implementaron diferentes estrategias de control para el seguimiento de trayectorias, incluyendo control por optimización y control por modo deslizante (SMC), evaluando su desempeño en términos de precisión y estabilidad. Asimismo, se diseñaron y fabricaron adaptadores personalizados tanto para los grippers Robotiq como para los dispositivos hápticos, permitiendo una interacción adecuada con los instrumentos quirúrgicos.

El sistema fue validado mediante pruebas experimentales sobre kits de entrenamiento médico, demostrando su capacidad para ejecutar tareas de sutura superficial de manera precisa, estable y repetible. Los resultados obtenidos evidencian el potencial del sistema como herramienta de asistencia o entrenamiento, y sientan las bases para futuras investigaciones orientadas a su uso en contextos clínicos reales. \\


\noindent \textbf{Palabras clave:}\
\noindent Teleoperación; Control háptico; Sutura superficial; Robot UR5; Control por optimización; Control por modo deslizante.
\input{secciones/abstract} 
\customchapter{INTRODUCCIÓN} 

\introsection{Presentación del tema de investigación}

Las heridas abiertas en la piel representan una de las principales causas de consulta en emergencias médicas. En 2021, en Estados Unidos, se registraron más de 722,000 visitas al área ambulatoria por cortes y heridas abiertas en extremidades superiores en hombres, mientras que Canadá reportó aproximadamente 151,000 visitas por lesiones similares entre marzo de 2023 y abril de 2024, lo que equivale a un promedio de 400 casos diarios \cite{Informe_emergencia_2021}\cite{CIHI}. Estas cifras reflejan una alta demanda, por lo que es crucial analizar su relevancia y los procedimientos medicos aplicables.

La sutura es un procedimiento médico esencial para el cierre adecuado de incisiones en tejidos blandos o heridas abiertas, tanto en cirugías como en atención ambulatoria \cite{Instru_quirurgica}. Su correcta ejecución no solo acelera el proceso de cicatrización, sino que también asegura el bienestar del paciente, previniendo complicaciones posteriores. Este procedimiento es ampliamente utilizado en medicina ambulatoria debido a la alta incidencia de heridas, como cortes irregulares o lesiones profundas ocasionadas por accidentes domésticos.

Al igual que en otros procedimientos quirúrgicos, las suturas implican riesgos que dependen de diversos factores, como la complejidad de la herida, las condiciones del entorno y la pericia del profesional a cargo. Estos riesgos incluyen infecciones, dehiscencias o necrosis si no se utilizan técnicas y materiales adecuados \cite{GONZALEZ-CELY2018}. El proceso requiere un monitoreo constante para mitigar complicaciones, incluso en entornos controlados, como salas de operaciones o consultorios. Es por ese motivo que en los últimos años, han surgido diversas innovaciones enfocadas en mejorar el proceso de sutura, como el desarrollo de hilos especializados, técnicas de cierre sin costura y el uso de fármacos que aceleran la cicatrización. 

Dentro de estas innovaciones se encuentra la implementación de secuencias apoyadas por elementos robóticos, a fin de incrementar el nivel de precisión en tareas quirugicas y controlar de mejor manera algunos factores de complejidad quirugica como el monitoreo constante o las complicaciones del entorno en las que se realizan las suturas. Ejemplos notables incluyen el robot Da Vinci y el sistema robóticob Zeus, que han permitido intervenciones quirúrgicas menos invasivas y con mayor precisión \cite{DaVinci}\cite{DaVinci_tumor_2024}. Robots diseñados para áreas específicas, como el NeuroMate para neurocirugía y el Robodoc para ortopedia, han sentado precedentes en el uso de tecnologías robóticas en suturas \cite{Historia_de_los_robots_cirujanos}. Estudios recientes han explorado modificaciones en robots como el IBIS para realizar suturas semiautomáticas, optimizando tiempos de manipulación y reduciendo riesgos asociados.

En conclusión, los avances tecnológicos en técnicas de sutura y la integración de la robótica abren nuevas posibilidades para mejorar los resultados clínicos, reducir complicaciones y acelerar los procesos de recuperación.

\introsection{Descripción de la situación problemática}


Las heridas abiertas, especialmente en extremidades, constituyen una causa recurrente de atención en emergencias hospitalarias. Según un estudio de 2018 realizado en cinco departamentos de emergencia en Estados Unidos, estas lesiones ocuparon el noveno lugar en la lista de motivos de consulta, con 2.7 millones de casos registrados \cite{Emergency_reasons}. Este alto volumen de atención representa un desafío significativo para el sistema de salud, particularmente en términos de tiempo requerido para el tratamiento, lo que puede generar sobrecarga operativa en emergencias.

El método más comúnmente empleado para tratar estas lesiones es la sutura, un procedimiento que, aunque efectivo, implica desgaste por parte del personal médico y técnico. La repetitividad inherente a la realización de suturas incrementa el riesgo de desarrollar trastornos músculo-esqueléticos en el personal médico cirujano o desarrollo prematuro de transtornos psicológicos como el síndrome burnout \cite{Burnout_España}\cite{Burnout_Peru}, presente tambien en el personal técnico. Hasta un 70\% de los cirujanos en procedimientos laparoscópicos han reportado síntomas de estos trastornos, lo que evidencia la carga física y mental asociada al trabajo repetitivo \cite{Desorden_Muscoesqueletico_Warren}. En entornos de emergencia, esta situación se agrava debido a limitaciones de espacio y posturas subóptimas, aumentando el riesgo de errores y complicaciones durante el procedimiento.

\begin{figure}[H]
    \centering
    \includegraphics[width=0.75\linewidth]{images/Estadistica2018.png}
    \caption{Ranking de los 20 casos más comunes en visitas al Área de Emergencias en 2018\cite{Emergency_reasons}}
    \label{fig:enter-label}
\end{figure}


Adicionalmente, el tiempo necesario para realizar suturas varía considerablemente según la complejidad de la herida y el área afectada \cite{Forsch}. En escenarios de alta demanda, esta variabilidad puede extender los tiempos de atención, reduciendo la eficiencia operativa de los servicios de emergencia. Este impacto no solo afecta la experiencia del paciente, sino también la capacidad del sistema para atender casos críticos con rapidez y efectividad.

En este contexto, resulta imperativo buscar soluciones tecnológicas que optimicen el proceso de sutura, mitiguen la carga física sobre los profesionales de la salud y reduzcan el tiempo de atención. La incorporación de herramientas robóticas y sistemas teleoperados se perfila como una estrategia prometedora para abordar estas problemáticas, mejorando tanto la eficiencia como la seguridad del procedimiento.
            
\introsection{Formulación del problema} 

La alta incidencia de heridas abiertas, especialmente en extremidades, constituye una de las principales causas de atención médica en emergencias, como lo reflejan las estadísticas del National Hospital Ambulatory Medical Care Survey (NHAMCS) \cite{Informe_emergencia_2021}. Estas heridas son tratadas, en su mayoría, mediante procedimientos manuales de sutura, los cuales presentan diversas limitaciones. Por un lado, la naturaleza repetitiva del procedimiento incrementa el riesgo de desarrollar trastornos músculo-esqueléticos en los profesionales de la salud, afectando hasta el 70 \% de los cirujanos que realizan procedimientos laparoscópicos \cite{Desorden_Muscoesqueletico_Warren}. Por otro lado, los tiempos de atención prolongados, derivados de la variabilidad técnica y la complejidad de las heridas, sobrecargan los servicios de emergencia, reduciendo su capacidad de respuesta ante otros casos críticos \cite{Informe_emergencia_2021}.

Adicionalmente, la precisión y calidad de la sutura manual dependen en gran medida de la experiencia del personal médico, lo que puede incrementar la probabilidad de complicaciones postoperatorias, como infecciones, dehiscencias y necrosis, en ausencia de materiales o técnicas adecuadas \cite{Instru_quirurgica}. Este panorama resalta la necesidad de innovaciones tecnológicas que optimicen el proceso de sutura, reduzcan los riesgos para los pacientes y mejoren las condiciones laborales del personal médico.

En este contexto, los avances en robótica quirúrgica y teleoperación han mostrado un gran potencial para abordar estas problemáticas. Sistemas como el robot Da Vinci han demostrado mejorar la precisión y eficiencia en procedimientos quirúrgicos complejos \cite{DaVinci}, mientras que investigaciones recientes han explorado el uso de manipuladores teleoperados con retroalimentación háptica para procedimientos de sutura automatizados, reduciendo los tiempos y riesgos asociados \cite{SIngle}. Sin embargo, aún existe un vacío en la implementación de soluciones específicamente diseñadas para optimizar la sutura en heridas superficiales de emergencia.

Por tanto, el problema central identificado es la necesidad de desarrollar e implementar un sistema robótico teleoperado con retroalimentación háptica que permita:
\begin{itemize}
    \item Mejorar la precisión y eficiencia del procedimiento de sutura.
    \item Reducir la carga física sobre los profesionales médicos.
    \item Minimizar las complicaciones asociadas al cierre manual de heridas.
\end{itemize} %aunque no se si esto se puwede confundir con objetivos

La solución a este problema puede transformar significativamente la atención médica en entornos de emergencia, garantizando una mayor seguridad para los pacientes y condiciones laborales óptimas para el personal de salud.













\introsection{Objetivos de investigación}

El objetivo del presente trabajo de tesis es integrar y controlar un prototipo de sistema robótico con módulo de teleoperación para la actividad de sutura superficial, haciendo uso de dos equipos hápticos y dos robots manipuladores.

%El objetivo principal del presente trabajo de tesis es desarrollar un prototipo de MóDULO DE CONTROL HAPTICO TELEOPERADO EN UN SISTEMA ROBÓTICO PARA REDUCCIóN DE RIESGO EN SUTURAS DE AREAS SUPERFICIALES.  

\introsection{Objetivos específicos}

\begin{itemize}
    \item {Diseño y ajuste del banco de pruebas que incluye terminales de sujeción adaptados al módulo háptico para manipulación en entornos de sutura, y adaptación del efector final de los dos brazos robóticos UR5 para poder hacer uso de la pinza laparoscópicas genéricas.} %tangible, porcentaje avance de diseño
    \item {Comparar arquitecturas de control (optimización, SMC, impedancia) para el seguimiento de trayectoria del UR5 en el espacio cartesiano.} %tangible, avance de diseño
    \item {Establecer la arquitectura maestro-esclavo entre el Geomagic Touch (USB) y el UR5 (Ethernet) mediante ROS 2, con control de los grippers Robotiq vía Modbus RTU.}%tangible, avance de diseño
    \item {Desarrollar un sistema de teleoperación maestro/esclavo entre el módulo háptico y el UR5.} %tangible, avance de diseño
    \item {Integrar los módulos de teleoperación, control y hardware para ejecutar tareas de sutura superficial en entorno controlado.} %tangible, avance de diseño


\end{itemize}

\introsection{Justificación}

El departamento de urgencias médicas recibe numerosos casos de pacientes con heridas abiertas, lo que obliga al personal médico a realizar suturas repetidamente a lo largo de su jornada. Estas actividades repetitivas pueden generar trastornos músculo-esqueléticos en los cirujanos, afectando su bienestar físico y, potencialmente, su rendimiento.

Con la implementación de un módulo háptico, se busca mejorar la eficiencia en la atención médica, ya que el robot manipulador asistirá al cirujano, reduciendo el margen de error y facilitando parte de las tareas repetitivas. De este modo, no solo será posible mitigar las molestias físicas que afectan al personal médico, sino también proporcionar un mayor nivel de seguridad a los pacientes durante el procedimiento.

Asimismo, se busca incorporar la retroalimentación háptica para que el cirujano, o el personal médico encargado de la sutura, pueda percibir el nivel de agarre o presión que experimenta el robot. Esto permitirá al profesional determinar con mayor precisión la fuerza adecuada para realizar la sutura de manera correcta y controlada. De este modo, se recreará la sensación de manipular instrumentos quirúrgicos reales, brindando al cirujano información táctil adicional que le ayudará a ejecutar las tareas con mayor exactitud y seguridad, ya que contará con más información aparte de la visual.


\introsection{Alcance y limitaciones / restricciones} % la sutura debe implicar casos de lab, las limitaciones tecnicas y biologicoas/medicas, hasta donde llegaremos con el proyecto. Limitaciones de $$, techno, alcance energético, al ser prototipo no se tendrá tanto cuidado en que plcas específicas son más econónomicas computaionlamnete.

Para el presente proyecto de tesis, se ha determinado el uso de los robots manipuladores UR5 de Universal Robotics, dado que estos se encuentran disponibles en la universidad. En consecuencia, no se realizará una comparación o evaluación de otros robots manipuladores en cuanto a su superioridad sobre el UR5 para el propósito de costura.

Asimismo, se utilizará el dispositivo háptico Touch de la empresa 3D Systems, con el propósito de modificarlo para adecuarlo a las necesidades de los cirujanos. Estas modificaciones, tanto mecánicas como electrónicas, no buscarán en esta etapa optimizaciones en términos de estética, eficiencia energética o computacional; en cambio, se priorizará el correcto funcionamiento del módulo háptico. Para el prototipado, se emplearán microcontroladores y sensores que permitan implementar las funciones básicas del sistema. Estos serán los disponibles en los laboratorios en el departamento de ingeniería mecatrónica y electrónica. En caso no estén disponibles se evaluará la compra acorde a la capacidad económica disponible de los tesistas.

El proyecto no será implementado en centros médicos. En su lugar, se evaluará su éxito utilizando kits de entrenamiento empleados por estudiantes de medicina para practicar suturas. Se tomará como consideración que las heridas a tratar en este proyecto son heridas tanto superficiales  y poco profundas; además, para facilidad de aplicación, la herida será del tipo limpia y de evolución complicada. 

El proyecto será considerado exitoso si el responsable logra realizar una sutura utilizando alguno de los procesos automatizados propuestos en el módulo. Cabe destacar que esta etapa del desarrollo no contempla pruebas con pacientes humanos, aunque se espera que en el futuro dicha aplicación sea factible.

%
\chapter{REVISIÓN CRÍTICA DE LA LITERATURA}

% Se realiza un análisis crítico de la literatura dirigida a robots quirúrgicos con el objetivo de identificar mecanismos, consideraciones técnicas y aplicaciones similares al procedimiento de suturas quirúrgicas. Asimismo, se investigan los distintos métodos de control utilizados para este tipo de sistemas hápticos, ya que, debido a la no linealidad del sistema y los requerimientos de estabilidad en su funcionalidad(movimiento, torque aplicado, seguimiento de trayectoria), no todos los métodos de control muestran un desempeño adecuado en la implementación.
% \cite{GONZALEZ-CELY2018}


%Sugerencia de cambio por la segunda revisión

En este capítulo se realiza una evaluación técnica de los antecedentes sobre robots quirúrgicos, con el objetivo de identificar mecanismos, configuraciones técnicas y aplicaciones relevantes al procedimiento de suturas. Este análisis se fundamenta en la revisión detallada de sistemas representativos que aporten bases concretas para el diseño metodológico y la interpretación de resultados en este proyecto.

Asimismo, se analizan los métodos de control aplicados a sistemas hápticos, evaluando su desempeño frente a las características no lineales del sistema y los requerimientos de estabilidad en torque, movimiento y seguimiento de trayectorias. Este enfoque permite identificar limitaciones técnicas y estrategias de control \cite{GONZALEZ-CELY2018} con potencial de adaptación al contexto de esta investigación.


\section{Antecedentes} %min 10 años de antiguedad,  solo hitos 

% Corrección avance2: le puse la ref que faltaba 

% La búsqueda de un sistema que disminuya los riesgos en una intervención quirúrgica ha permitido el desarrollo de diversos sistemas, técnicas y tecnologías a lo largo de la historia médica. Es bajo esta búsqueda que la robótica aparece como solución y se convierte en un tópico que ha resonado en las últimas décadas por las posibilidades que su desarrollo ofrece en el campo de las intervenciones no invasivas y de precisión. 
% La investigación en robótica para trabajos en laparoscopia es uno de los pilares dentro de los avances en cirugías no invasivas, tal y como se menciona en \cite{Historia_de_los_robots_cirujanos} donde menciona hasta un incremento de 10 veces dentro del campo de las cirugías laparoscópicas. Investigaciones como las realizadas con el robot DaVinci \cite{ma_da_2024} han mostrado resultados alentadores y han impulsado al desarrollo de nuevos modelos de robots y nuevas técnicas para la aplicación de sistemas de control, como el control por impedancia. Por ejemplo, un último hito fue la realización exitosa de una cirugía para la extracción de un tumor benigno en China \cite{ma_da_2024}, donde el uso del robot DaVinci fue crucial para evitar la sobremanipulación del área a operar. 

%Asimismo, a la par del desarrollo de sistemas robóticos en cirugía se puede encontrar el desarrollo de la manipulación de los mismos, la teleoperación. La robótica en cirugía a lo largo de las distintas aplicaciones y modelos desarrollados busca perfeccionar la 

% Propuesta de corrección:

La búsqueda de sistemas que reduzcan los riesgos durante intervenciones quirúrgicas ha impulsado el desarrollo de diversas técnicas y tecnologías a lo largo de la historia médica. En este contexto, la robótica ha emergido como una solución clave, destacándose en las últimas décadas por su potencial para optimizar procedimientos quirúrgicos no invasivos y de alta precisión

Según \cite{Historia_de_los_robots_cirujanos}, la integración de robots en este campo ha llevado a un aumento de hasta 10 veces en la precisión de las intervenciones laparoscópicas. Un caso destacado es el robot Da Vinci, cuya investigación y desarrollo han dado lugar a técnicas innovadoras como el control por impedancia, diseñado para ajustar dinámicamente la fuerza aplicada por los manipuladores robóticos en respuesta a las interacciones con el tejido. De esta forma, se evitaría traspasar o dañar los tejidos por la falta de este método de control. 

% FAlta corregir la parte que pregunta si es una investigación, asúmo que con decir un poco más de la fuente basta pero no estoy seguro uu, borrar "podemos"





\section{Estado del Arte}

\subsection{Sistemas robóticos teleoperados}
La teleoperación permite la manipulación del movimiento de un robot sin la necesidad de centrarse en el control interno del mismo \cite{you_assisted_2012}, enfatizando el trabajo del control a un nivel más alto y permitiendo realizar tareas más complejas a través del sistema Maestro-Esclavo.
Dentro de las investigaciones de teleoperación para robots podemos encontrar el trabajo en la obtención del movimiento del operario, siendo la correcta detección de esta un punto crucial al momento de traducir un movimiento natural en ángulos de rotación de un brazo robótico. Una investigación realizada para el simposio internacional de Electrónica en 2019 testeó el control de un brazo de 4 grados de libertad  mediante la herramienta Kinetic desarrollado por Microsoft \cite{syakir_teleoperation_2019} para reemplazar el sistema Maestro convencional. 



% cirujanos podemos encontrar los modelos comerciales, en los que encontramos al robot DaVinci, Zeus, Hugo RAS \cite{prata_state_2023} 

\subsection{Sistemas robóticos aplicados a cirugías}
La inclusión de robots ha iniciado una nueva era en campo médico, especialmente orientado a cirugías, ya que estos permiten un control de un manipulador que permite que el médico cirujano no esté necesariamente presente en el lugar de la operación. Asimismo, estos se han adaptado para reducir su tamaño, de tal forma, que no es necesario realizar cortes profundos en para una operación, sino que gracias a esta modificación el brazo robótico cuenta con una extensión suficientemente larga para llegar al lugar deseado para la operación. Asimismo, en cirugías laparoscópicas, por ejemplo, usualmente cuenta con cámaras que le permiten al médico obtener información sobre el entorno sobre el que va a operar. Robots como el Davinci fueron específicamente diseñados para su uso en entornos de operaciones y, a pesar de su elevado costo. Muchos hospitales tomaron la decisión de incluirlos durante sus operaciones, ya que además de minimizar el tamaño de la incisión, ayudan a mejorar la precisión de los movimientos del cirujano \cite{review_of_haptic_feedback}. 





\subsection{Reconocimiento y visión computacional en cirugías}
Desde el descubrimiento de las imágenes de rayos X para el escaneo del cuerpo humano desde 1895 que se usaron para un diagnóstico médico se comenzó a requerir un análisis de estas imágenes y con el avance de la tecnología, surgieron numerosos exámenes que resultaban en la imagen sobre alguna parte del cuerpo humano. El análisis de estas imágenes era de suma importancia, ya que ayudarían a reconocer la presencia de alguna herida u objeto maligno en el cuerpo humano. Sin embargo, ello también dio paso a numerosas investigaciones sobre el análisis de estas imágenes con un enfoque computacional. En este ámbito machine learning no resulta de mucha ayuda, ya que métodos basados en regresiones lineales o árboles de decisión no resultan suficientes. Por ello, a las imágenes médicas se aplica mejor algoritmos con redes convulucionales que permiten la extracción de las características de las imágenes que usualmente no tiene gran calidad. De esta forma, inclusive, se puede buscar detectar la aparición de enfermedades antes de mostrar los signos de esta sobre el cuerpo humano. Estos algoritmos necesitan de una gran cantidad de bases de datos para aprender de estas y realizar las detecciones con mayor precisión, ya que con él cada imagen que analiza no deja de aprender sobre sus errores y continúa mejorando. Por ello, es posible esperar que su influencia en el ámbito médico llegue a causar un gran impacto.


\subsection{Inclusión de la tecnología Háptica}


La tecnología háptica se define como la disciplina que combina la percepción sensorial del tacto con el control de la respuesta a este estímulo, permitiendo diversas aplicaciones tecnológicas \cite{Informacion_de_aplicaciones_tecnologia_haptica}. Esta tecnología busca transmitir las sensaciones físicas al usuario mediante la información recuperada de un modelo vitural o sensores, recreando experiencias que emulan la percepción directa como si el usuario interactuara presencialmente con el entorno. Su aplicación se ha destacado principalmente en la simulación de entornos virtuales para el entrenamiento del el uso de algún dispositivo como en la conducción de vehículos \cite{Ejemplo_haptic_manejo_carro}, manejo de aeronaves \cite{Ejemplo_haptic_piloto_avion} y en el área de capacitación en la industria espacial \cite{Ejemplo_haptic_entrenamiento_espacial}.


En el ámbito médico, las operaciones requieren la plena atención del médico, especialmente en los aspectos visual y háptico, lo que permite un buen desempeño para realizar correctamente la actividad. Asimismo, la robótica se ha introducido en este sector, como es el caso del robot RIO System, que cuenta con retroalimentación háptica y se utiliza en cirugías ortopédicas, o el robot ALF-X, orientado a procedimientos RMIS (cirugía mínimamente invasiva asistida por robot) \cite{robotica_haptics_retos_beneficios}. De este modo, aunque los robots manipuladores para cirugía comenzaron siendo únicamente teleoperados, muchos están incorporando la retroalimentación háptica debido al desempeño que esta proporciona. Al tratar con tejido blando de los órganos, la fuerza que se aplica con el robot debe ser precisa, ya que es posible dañar estos tejidos si no se controla adecuadamente. Solo con la retroalimentación visual podrían ocurrir errores al estimar la profundidad. Esto se refleja en la referencia, donde se menciona que, al incluir la retroalimentación háptica, los daños generados en un simulador de operaciones se redujeron en un 55% \cite{robotica_haptics_retos_beneficios}.


\chapter{MARCO TEÓRICO}

En el presente capítulo se realizará una exploración de los conceptos necesarios a fin de desarrollar el módulo de teleoperación con feedback háptico. En primer lugar, se aborda un acercamiento al ámbito médico dentro del campo de la robótica en cirugías. Asimismo, se explica la estructura de un sistema háptico y las distintas técnicas de aplicación existentes. Por otro lado, se detallan las teorías de diseño cinemático y dinámico en un manipulador, como es el caso del robot UR5. Además, se describe el método de control aplicado en el sistema y las ecuaciones necesarias para su implementación. Por último, se presentan las herramientas de detección y prevención de colisiones en un entorno de trabajo con dos robots manipuladores, junto con la herramienta de visión computacional.


\section{Suturas externas}
Una herida cutánea o externa, desde su generación hasta su cierre, presenta 3 etapas previas a su curación. La Fase inflamatoria, donde se empieza la coagulación de la herida; la fase proliferativa, donde se crean nuevos tejidos que cubrirán la herida, y la fase de remodelación, donde la heridad termina de cicatrizar \cite{Cicatritacion}. El tiempo de duración de cada etapa dependerá de la naturaleza de la herida y su condición al momento de la revisión. Tomando estos factores en cuenta se puede determinarán si es necesario asegurar la conclusión de alguna de las fases mencionadas con una sutura o si, por el contrario, solo desinfectar la zona afectada.

Para los casos de aplicación de suturas, es considerada en el campo médico como el material encargado del cierre de heridas\cite{GONZALEZ-CELY2018}. De este modo, las suturas enfocadas en heridas superficiales se considerarían como suturas externas. El proceso de suturar es el acercamiento de tejidos separados productos de un corte para su correcta cicatrización. Tomando en cuenta el área afectada, puede necesitarse hilos especiales  e instrumental quirúrgico adicional \cite{GONZALEZ-CELY2018}.

    \subsection{Herramientas comunes}
    Dentro de las herramientas básicas para realizar una sutura se encuentra el porta agujas, la pinza, tijeras y el bisturí. Además de estos, se debe complementar con anestésico local.
    \begin{figure}[H]
        \centering
        \includegraphics[width=0.75\linewidth]{images/anastesicos.png}
        \caption{Anestésicos locales más utilizados\cite{GONZALEZ-CELY2018}}
        \label{fig:enter-label}
    \end{figure}
    \subsection{Características a considerar}
    Dentro de las características más importantes, está el material de la sutura. Como se describió a principios del capítulo, la sutura también es la denominación del material con el que se realiza el cierre de la herida; pudiendo ser hilos, grapas, bandas adhesivas, etc. Su elección dependerá del tiempo de permanencia requerido y del área a suturar. 

    Para el caso del uso de hilos, el calibre y el tiempo de permanencia del hilo utilizado variará dependiendo del área anatómica a suturar, por ejemplo, una área de alto movimiento, como una articulación, necesitará un tiempo más prolongado. Asimismo, el material del hilo variará dependiendo de la aplicación y condiciones de la piel. 

    \begin{figure}[H]
        \centering
        \includegraphics[width=1\linewidth]{images/hilos.png}
        \caption{Tipos de hilos en suturas\cite{GONZALEZ-CELY2018}}
        \label{fig:enter-label}
    \end{figure}

    
    \subsection{Técnicas de aplicación de suturas}
    \textbf{Pasos para realizar una sutura}:
    \begin{enumerate}
        \item Limpieza y desinfección de la herida.
        \item Aplicación anestesia local.
        \item Cierre de la herida con sutura de hilo.
        \item Nudo para con doble lazada y luego lazadas simples\cite{GONZALEZ-CELY2018} 
        \item Cubrir área suturada.
    \end{enumerate}



% %explicar la teoria presente en libros
% \section{Método de sutura robótica} %metodologia

% La sutura realizada por robots es crítico al momento de dar por finalizado un proceso quirurgico, su aplicación en laparoscopia 

\section{Definición de los sistemas Háptico}
  

\subsection{Principios de Funcionamiento de la Retroalimentación Háptica en Robótica}

El sentido del tacto es uno de los más importantes para la extracción de características físicas de los objetos que se encuentran en el entorno y se manipulan, ya que permite identificar la dureza y la rugosidad \cite{El_sentido_del_tacto_revision}. De esta forma, combinando la retroalimentación de los otros sentidos, el ser humano es capaz de comprender lo que está manipulando. Es por esto que la parte háptica es necesaria para mejorar la capacidad de acción del ser humano \cite{spence_gallace_touch}. Dentro del ámbito de la sutura, es muy importante la noción de las características físicas del tejido que se está manipulando, ya que de ello dependerá que el método aplique más o menos fuerza al dispositivo de control para realizar la incisión de la aguja o el agarre de la piel para mantener estable el área que se desea perforar. En caso contrario, existe un alto riesgo de dañar los tejidos únicamente basándose en la vista, lo que podría llevar a consecuencias fatales para el paciente durante la operación.

% De este modo, el dispositivo háptico que se utilizará para esta investigación será capaz de transmitir únicamente la información física relacionada con la rigidez del objeto, es decir, la fuerza que opone resistencia del objeto a ser deformado. De esta forma, para el dispositivo háptico deberá ser capaz de transmitir fuerza en un espacio tridimensional. De está forma, el medico será capaz de recibir una retroalimentación de rigidez en cualquier dirección lo que mejorará su acionar al evitar ejercer más fuerza de la debida. Asímismo, el dispositivo transmitirá la fuerza por consecuencia del control de motores internos al sistema de control que usará el cirujano, ya que se buscará que estos imiten la rigidez que se está sensando al final del brazo manipulador. 





\subsection{Caracterización Técnica del Geomagic Touch}


\begin{figure}[H]
    \centering
    \includegraphics[width=0.75\linewidth]{encabezados/geomagic_touch.png}
    \caption{Dispositivo háptico Geomagic touch de 3D Systems}
    \label{fig:Geomagic_touch_base}
\end{figure}


El dispositivo háptico utilizado para el trabajo de investigación es el Geomagic Touch de la empresa 3D Systems. Este dispositivo cuenta con 6 grados de libertad. De entre estos, los 6 son leidos con encoders y solamente 3 son actuados. Este dispositivo permite al usuario recibir una retroalimentación por las fuerzas que ejercen en cada una de sus 3 articulaciones. De esta forma, será necesario estimar la posición y orientación del gripper y controlar la velocidad máxima de los motores para simular el estímulo háptico. Asímismo, se tendrán hasta 3.3N de fuerza nominal para proyectar el estimulo de la rigidez sobre el usuraio al mando del sistema de control. Ello será sufiente, dado que la superficie a cortar de los tejidos no no debería superar esta fuerza al usarse una aguja afilada. Sin embargo, se estima que el rago de la fuerza de acionamiento pueda no ser suficiente, dada la extención que podría requedrir la manipulación del hilo y la aguja para suturar un área deseadas. De esta forma, el presente trabajo se limitará a su aplicación a heridas abiertas que no requieran salir del rango de operación para la reatroalimentación de fuerza por parte del dispositivo. Asímismo, el dipositivo podrá comunicarse a 1000 Hz lo cual será un componente determinante para determinar el tiempo a la que se podrá actulaizar la ley de control para los motores.


\begin{figure}[H]
    \centering
    \includegraphics[width=0.75\linewidth]{images/technical_geomagic_touch.png}
    \caption{Especificaciones técnicas del dispositivo háptico Geomagic touch de 3D Systems}
    \label{fig:enter-label}
\end{figure}

\subsection{Interfaz Háptica-Robótica para Procedimientos Médicos}
Para enlazar el sistema háptico se utilizará el software que brindra la empresa 3D Systems para el sistema operativo Linux. Este software OpenHaptics permitirá desarrollar el controlador en C/C++. De esta forma, como tambien se planea controlar los robots manipuladores en este sistema operativo, se buscará realizar el intercambio de información mediante la publicación y subscripción de nodos en ROS. Además, teniendo en cuenta que el robot manipulador será teleoperado se buscará principalmente controlar la parte del efector final con la punta del lapiz que cuenta el dispositivo háptico, ya que la forma del controlador y el brazo no son similares. Sin embargo, se buscará que llegen a tener posiciones similares para que el moviento sea más fluido. Asímismo, se buscará implementar una interfaz gráfica donde se pueda observar el moviento del robot en tiempo real con Rviz.  Estas tareas serán silimlares a a las vistas en \cite{teleoperation_irb120_geomagic}, donde se realizó un proceso similar para un robot ABB IRB 120. 


\subsection{Adaptación de los Manipuladores para Control de Sutura mediante Háptica}

El dispositivo contará con una adaptación específica para el procedimiento de sutura, ya que este proceso requiere la sujeción de una aguja e hilo. Por lo tanto, se adaptará el dispositivo háptico para cumplir con estos requisitos. Dado que el Geomagic Touch se puede desacoplar, se diseñará un adaptador en forma de lápiz que se acoplará al efector final para desempeñar esta función. Este diseño se basará en un mecanismo similar al presentado en \cite{arm_pivot_joints_stiffness}, donde se logró adaptar un sistema que permite la inserción de los dedos para evaluar el efecto de un punto pivote sobre la rigidez que se mide.

\begin{figure}[H]
    \centering
    \includegraphics[width=0.75\linewidth]{images/geomagic_adaptation_example.png}
    \caption{Modificación al efector final del módulo geomagic touch en \cite{arm_pivot_joints_stiffness}}
    \label{fig:enter-label}
\end{figure}

   
\section{Modelo Cinemático del robot serial}

Un robot serial puede representarse como la unión cinemática de distintos cuerpos rígidos conectados por uniones rotativas o de deslizamiento \cite{siciliano_robotics_2009}. Para lograr un control óptimo del efector final del robot, es fundamental comprender cada uno de los elementos presentes en su análisis cinemático. Esto se debe a que el efector final responderá al movimiento generado por la acción realizada en cada extremidad del robot.

La cinemática permite expresar este análisis mediante expresiones matemáticas en función del movimiento de las articulaciones respecto a un punto de referencia, que generalmente es la base del robot. Este análisis se puede desglosar en subtemas según su finalidad: la Cinemática Directa, que calcula la posición y orientación del efector final a partir de los ángulos de rotación de cada articulación; y la Cinemática Inversa, que determina los ángulos de rotación en cada articulación necesarios para posicionar el efector final en coordenadas y orientación deseadas.

    \subsection{Matriz de Transformación}

    Para poder representar correctamente el movimiento de una articulación del robot, debemos considerarlo como un cuerpo rígido; es decir, que no presenta deformación; además que puede ser representado espacialmente por su posición y orientación\cite{siciliano_robotics_2009}. Es dicha representación la que podemos definirla bajo un conjunto de transformaciones tanto de rotación como de traslación.
    \begin{figure}[H]
        \centering
        \includegraphics[width=0.95\linewidth]{images/pos_or.png}
        \caption{Posición y orientación de un cuerpo rígido}
        \label{fig:por_or}
    \end{figure}
    
    Podemos expresar el punto P en los sistemas de referencia $A$ y $B$: 
    
    $
    \begin{cases}
        ^{A}P = ^{A}x\boldsymbol{\hat{x_A}}  + ^{A}y\boldsymbol{\hat{y_A}}+^{A}z\boldsymbol{\hat{z_A}} \\
        ^{B}P = ^{B}x\boldsymbol{\hat{x_B}}  + ^{B}y\boldsymbol{\hat{y_B}}+^{B}z\boldsymbol{\hat{z_B}}
    \end{cases}
    $

    Si proyectamos el punto en $B$ respecto a $A$ podemos realizarlo mediante la operación vectorial producto punto, lo que resulta en:
    
    $
    \begin{cases}
        ^{A}x = ^{B}p \cdot \boldsymbol{\hat{x_A}}\\
        ^{A}y = ^{B}p \cdot \boldsymbol{\hat{y_A}}\\
        ^{A}z = ^{B}p \cdot \boldsymbol{\hat{z_A}}
    \end{cases}
    $

    
    $
    \begin{cases}
        ^{A}x = ^{B}x(\boldsymbol{\hat{x_B}} \cdot \boldsymbol{\hat{x_A}})  + ^{B}y(\boldsymbol{\hat{y_B}} \cdot \boldsymbol{\hat{x_A}})+^{B}z(\boldsymbol{\hat{z_B}} \cdot \boldsymbol{\hat{x_A}})\\
        ^{A}y = ^{B}x(\boldsymbol{\hat{x_B}} \cdot \boldsymbol{\hat{y_A}})  + ^{B}y(\boldsymbol{\hat{y_B}} \cdot \boldsymbol{\hat{y_A}})+^{B}z(\boldsymbol{\hat{z_B}} \cdot \boldsymbol{\hat{y_A}})\\
        ^{A}z = ^{B}x(\boldsymbol{\hat{x_B}} \cdot \boldsymbol{\hat{z_A}})  + ^{B}y(\boldsymbol{\hat{y_B}} \cdot \boldsymbol{\hat{z_A}})+^{B}z(\boldsymbol{\hat{z_B}} \cdot \boldsymbol{\hat{z_A}})
    \end{cases}
    $

    Vectorialmente, podemos expresarlo:
    \begin{equation*}
        \begin{bmatrix}
            ^{A}x\\^{A}y\\^{A}z 
        \end{bmatrix} = 
        \begin{bmatrix}
            \boldsymbol{\hat{x_B}} \cdot \boldsymbol{\hat{x_A}} & \boldsymbol{\hat{y_B}} \cdot \boldsymbol{\hat{x_A}} & \boldsymbol{\hat{z_B}} \cdot \boldsymbol{\hat{x_A}} \\
            \boldsymbol{\hat{x_B}} \cdot \boldsymbol{\hat{y_A}} & \boldsymbol{\hat{y_B}} \cdot \boldsymbol{\hat{y_A}} & \boldsymbol{\hat{z_B}} \cdot \boldsymbol{\hat{y_A}} \\
            \boldsymbol{\hat{y_B}} \cdot \boldsymbol{\hat{z_A}} & \boldsymbol{\hat{y_B}} \cdot \boldsymbol{\hat{z_A}} & \boldsymbol{\hat{z_B}} \cdot \boldsymbol{\hat{z_A}}
        \end{bmatrix} \begin{bmatrix}
            ^{B}x \\  ^{B}y \\ ^{B}z
        \end{bmatrix}
    \end{equation*}

    Operando se consigue la expresión matricial correspondiente a la matriz de rotación, esta se define como $^{A}R_B$ \cite{siciliano_robotics_2009}. Si analizamos las rotaciones realizadas en los 3 ejes principales podemos definir 3 matrices elementales de rotación, que son de utilidad al momento de expresar transformaciones en la orientación. 

    \begin{equation}
        R_x(\theta) = 
        \begin{bmatrix}
            \cos{\theta} & -\sin{\theta} & 0\\
            \sin{\theta} & \cos{\theta} & 0\\
            0 & 0 & 1
        \end{bmatrix}
    \end{equation}
    \begin{equation}
        R_y(\theta) = 
        \begin{bmatrix}
            \cos{\theta} & 0 & \sin{\theta} \\
            0 & 1 & 0 \\
            -\sin{\theta} & 0 & \cos{\theta}            
        \end{bmatrix}
    \end{equation}
    \begin{equation}
        R_z(\theta) = 
        \begin{bmatrix}
            1 & 0 & 0\\
            0 & \cos{\theta} & -\sin{\theta}\\
            0 & \sin{\theta} & \cos{\theta}             
        \end{bmatrix}
    \end{equation}

    Para representar una traslación del eje $B$ respecto al eje veremos que se puede traducir en una suma vectorial de la posición del sistema $A$ y el vector $^{A}t_B$
    
 \begin{equation*}
     ^{A}P  = ^{A}t_B +  ^{B}P
 \end{equation*}

 Si juntamos tanto traslación como rotación obtenemos 
 \begin{equation*}
     ^{A}P  = ^{A}t_B +  ^{A}R_B \ ^{B}P
 \end{equation*}

 Matricialmente, podemos reducir la expresión a:
    \begin{equation}        
        \begin{bmatrix}
            ^{A}P\\1  
        \end{bmatrix} = \begin{bmatrix}
            ^{A}R_B & ^{A}t_B \\
             0 & 1
        \end{bmatrix} \begin{bmatrix}
            ^{B}P \\ 1
        \end{bmatrix}
    \end{equation}

 Entonces la matriz resultante representará la matriz de transformación homogénea $^{A}T_B$
    
    \subsection{Cinemática Directa}
    \begin{figure}[H]
        \centering
        \includegraphics[width=1\linewidth]{images/graf2.png}
        \caption{Transformación de coordenadas en una Cadena cinemática abierta \cite{siciliano_robotics_2009}}
        \label{fig:enter-label}
    \end{figure}
    La Cinemática Directa permite encontrar la posición y orientación del efector final, tomando como entrada las transformaciones realizadas en cada una de las extremidades, una después de la otra. Esta cadena de transformaciones homogéneas se puede expresar como una multiplicación de cada matriz de transformación aplicada a las extremidades que representa finalmente la matriz de transformación homogénea general desde la base del robot o sistema de referencia inercial hasta el efector final.

    \begin{equation} \label{t-homog}
        ^{0}T_n = (^{0}T_1)(^{1}T_2)...(^{n-1}T_n)
    \end{equation}

    \subsection{Método Denavit-Hartenberg}

    Este método permite, de manera sistemática, obtener las matrices homogéneas de cada extremidad mediante un conjunto de pasos tomando en cuenta la extremidad anterior.
    \begin{figure}[H]
        \centering
        \includegraphics[width=\linewidth]{images/DH_general.png}
        \caption{Método Denavit-Hartemberg}
        \label{fig:enter-label}
    \end{figure}

   \textbf{Pasos para la elección de los ejes:}
    \begin{itemize}
        \item Eje $z_i$: Se ubica $z_i$ sobre el eje de movimiento de la articulación siguiente. 
        \item El origen de coordenadas se ubica en la intersección de los ejes $z_{i-1}$ y $z_i$. Si los ejes son paralelos se ubica en la intersección de $z_i$ con la normal en común entre $z_{i-1}$ y $z_i$.
        \item Eje $x_i$: Se ubica en dirección de $z_{i-1}$ x $z_i$. Si los ejes son paralelos se ubica en la normal en común entre $z_{i-1}$ y $z_i$.
        \item Eje $y_i$: Se ubica completando el eje coordenado.
        
    \end{itemize}

    \textbf{Parámetros a identificar }
    \begin{itemize}
        \item $\theta_i$: Rotación del eje $x_{i-1}$ a $x_i$ alrededor del eje $z_{i-1}$
        \item $d_i$: Distancia del sistema coordenado $i-1$ a $i$ a lo largo del eje $z_{i-1}$ hasta la intersección de $x_i$ con $z_{i-1}$
        \item $a_i$: Distancia de la intersección de $x_i$ con $z_{i-1}$  a lo largo del eje $x_{i_1}$ hasta sistema coordenado $i$
        \item $\alpha_i$: Rotación del eje $z_{i-1}$ a $z_i$ alrededor del eje $x_{i}$
    \end{itemize}

    Una vez identificado los parámetros para cada sistema, los ubicamos en un cuadro:
    \\
    
    

    \begin{table}[H]
        \begin{center}
        \begin{tabular}{| c | c | c | c | c |}
            \hline
            i & $d_i$ & $\theta_i$ & $a_i$ & $\alpha_i$ \\ \hline
            1 & $d_1$ & $\theta_1$ + $q_1$ & $a_1$& $\alpha_1$ \\ \hline
            2 & $d_2$ & $\theta_2$ + $q_2$ & $a_2$ & $\alpha_2$ \\ \hline
             & &... & &\\ \hline
            n & $d_n$ & $\theta_i$ + $q_n$ & $a_n$ & $\alpha_n$ \\ \hline
        \end{tabular}
            \caption{Tabla con parámetros}
            \label{tab:DH}
        \end{center}
    \end{table}

    
    \begin{table}[H]
        \begin{center}
        \begin{tabular}{| c | c | c | c | c |}
            \hline
            i & $d_i$ & $\theta_i$ & $a_i$ & $\alpha_i$ \\ \hline
            1 & $0.1625$ & $q_1$      & $0$& $-\frac{\pi}{2}$ \\ \hline
            2 & $0$      & $q_2$      & $0.425$ & $0$ \\ \hline
            3 & $0$      & $q_3$       & $0.3922$ & $0$ \\ \hline
            4 & $0.1333$ &$q_4 - \pi$  & $0$ & $\frac{\pi}{2}$ \\ \hline
            5 & $0.0997$ & $q_5+\pi$ & $0$ & $\frac{\pi}{2}$ \\ \hline
            6 & $0.0996$ & $q_6$     & $0$ & $0$ \\ \hline
        \end{tabular}
            \caption{Tabla de parámetros Denavith Hartenberg}
            \label{tab:DH}
        \end{center}
    \end{table}


    Para llevar los parámetros a una Transformación homogénea se toma este orden de para cada movimiento realizado:
    \begin{enumerate}
        \item Rotación $\theta_i$ en el eje $z_{i-1}$
        \item Traslación $d_i$ en el eje $z_{i-1}$
        \item Traslación $a_i$ en el eje $x_{i}$
        \item Rotación $\alpha_i$ en el eje $z_{i}$
    \end{enumerate}
    Multiplicando cada Matriz homogenea se consigue:

    \begin{equation*}
        ^{i-1}T_{i}(\theta_i,d_i,a_i,\alpha_i)= 
        \begin{bmatrix}
            \cos{\theta_i}& -\cos{\alpha_i}\sin{\theta_i} &\sin{\alpha_i}\sin{\theta_i} & a_i\cos{\theta_i} \\
            \sin{\theta_i}& \cos{\alpha_i}\cos{\theta_i} &-\sin{\alpha_i}\cos{\theta_i} & a_i\sin{\theta_i} \\
            0 & \sin{\alpha_i} & \cos{\alpha_i} & d_i \\
            0 & 0 & 0 & 1
        \end{bmatrix}
    \end{equation*}

    La ecuación \ref{t-homog} permite obtener la representación de la posición y orientación del efector final respecto a la base del robot.
       
    
    \subsection{Cinemática Inversa}

    La cinemática inversa permite calcular las configuraciones articulares de cada elemento en la cadena de articulaciones necesario para lograr una posición y orientación deseada.
    El cálculo de la cinemática inversa en robots como el UR5 de 6 grados de libertad resulta tedioso bajo la directriz de métodos de resolución de ecuaciones clásicos, puesto que se encuentran con expresiones no lineales incompatibles a la resolución por sistema de ecuaciones. Por otro lado, existe el método numérico; este forma de resolución es de naturaleza recursiva y hace uso de técnicas como la minimización de una función costo del error respecto a la posición actual y la posicion deseada del efector final.

    El método numérico hace uso de la matriz Jacobiana para su cálculo, ya sea considerando su inversa o su transpuesta.    
    
    \textbf{Matriz Jacobiana}
    \begin{equation}
        \boldsymbol{J} = \frac{\partial f(q)}{\partial q}
    \end{equation}

    El método numérico propuesto es La Resolución por Método de Gradiente, donde se toma una función costo dependiente de los valores articulares del robot, cuya gradiente permitirá encontrar valores para cada articulación acercando la posición del efector final a la posición deseada.

    \textbf{Método Gradiente}\\
    De la cinemática directa:    
    \begin{equation*}
        \boldsymbol{x}_d = f(\boldsymbol{q}) \longrightarrow \boldsymbol{x}_d - f(\boldsymbol{q}) = 0            
    \end{equation*}
    Donde la expresión resultante es el error respecto a la posición deseada. Con esta expresión podemos definir la función costo $g(\boldsymbol{q})$
    \begin{equation*}
        g(\boldsymbol{q}) = \frac{1}{2} \|\boldsymbol{x}_d - f(\boldsymbol{q})\|^2
    \end{equation*}

    Se calcula el gradiente de $g(\boldsymbol{q})$:
    \begin{equation*}
        \nabla g(\boldsymbol{q}) = -(\frac{\partial f(\boldsymbol{q})}{\partial \boldsymbol{q}})^T (\boldsymbol{x}_d - f(\boldsymbol{q}))
    \end{equation*}

    Finalmente, se puede aplicar el método gradiente de manera recursiva mediante un parámetro $\alpha$ que determinará el tamaño de paso en el avance del gradiente hasta encontrar los valores articulares adecuados.
    \begin{equation}
        \boldsymbol{q}_{k+1} = \boldsymbol{q}_k + \alpha\boldsymbol{J}^T (\boldsymbol{q}_k)(\boldsymbol{x}_d-f(\boldsymbol{q}_k))
    \end{equation}
    \textbf{Método por Optimización Cuadrática}\\
    Partiendo de la función costo $g(\boldsymbol{q})$ se define un problema de minimización, con $e$ como el error de posición, de la forma:    
    \begin{mini*}|s|[0]
        {\boldsymbol{\dot{q}}}{\|\boldsymbol{J(q)}\boldsymbol{\dot{q}} - \boldsymbol{e}\|^2}
        {}
        {\label{eq:minimizationProblem1}}
    \end{mini*}
    
    Expandiendo y ordenando podemos plantear un problema cuadrático de la forma: 
    \begin{mini}|s|[0]
        {\boldsymbol{\dot{q}}}{\frac{1}{2}\boldsymbol{\dot{q}}^{T}\boldsymbol{H} + \nabla g(\boldsymbol{q})\boldsymbol{\dot{q}}}
        {}
        {\label{eq:minimizationProblem}}
    \end{mini}
    donde $\boldsymbol{H} =\boldsymbol{J(q)}^{T}\boldsymbol{J(q)} $, como matriz Hesiana y el gradiente $\nabla g(\boldsymbol{q}) = -(\boldsymbol{J(q)}^{T}\boldsymbol{e})^{T}$ 
    % \\

    % $\boldsymbol{q}$ : configuración articular \\
    % $\boldsymbol{\dot{q}}$: velocidad articular \\
    % $\boldsymbol{J(q)}$ : Matriz Jacobiana \\
    % $\boldsymbol{e}$ : error de posición \\
    % $\boldsymbol{H}$ : Matriz Hesiana\\
    % $\nabla g(\boldsymbol{q})$ : Gradiente
    
\section{Modelo Dinámico del robot serial }

El modelo dinámico permite relacionar el movimiento del robot mediante el análisis de la fuerza y torques aplicados en cada articulación. Mediante este análisis se pueden determinar con mayor fiabilidad un sistema de simulación del robot y el modelamiento de un sistema de control adecuado para entornos reales. Sin embargo, encontrar el modelo dinámico de un robot manipulador implica poder determinar variables matriciales que gobiernan el comportamiento del robot, como la matriz de inercia. Ante este problema, existen métodos matemáticos para el diseño de modelos dinámicos, sea de manera analítica, mediante ecuaciones simbólicas; o de manera numérica, mediante recursividad. 

    \subsection{Newton Euler}    
    Este método es de naturaleza recursiva, puesto que los valores se operarán con los obtenidos anteriormente. Es esta naturaleza la que permite obtener la dinámica inversa, donde se consiguen las fuerzas y torques necesarios para lograr un movimiento deseado. Además, permite establecer un algoritmo que, sumado a herramientas computacionales, logra realizar el cálculo con mayor rapidez. 

    El método Newton-Euler realiza un análisis desde la formulación de la $2^{da}$ ley de Newton para traslación y la $2^{da}$ ley de Euler para la rotación.
    \begin{equation*}
        \boldsymbol{f}_i = \boldsymbol{f}_{i+1} + m_i\ddot{\boldsymbol{p}}_{c_i}-m_i\boldsymbol{g}_0
    \end{equation*}
    \begin{equation*}
        \boldsymbol{\mu}_i = \boldsymbol{\mu}_{i+1} - \boldsymbol{f}_{i}\times\boldsymbol{r_{i-1,c_i}}+\boldsymbol{f}_{i+1}\times\boldsymbol{r_{i,c_i}}+\times\boldsymbol{I}_i\times\dot{\boldsymbol{\omega}}_i+\boldsymbol{\omega}_i\times\boldsymbol{I}_i\boldsymbol{\omega}_i
    \end{equation*}

    Para un eslabón $i$ podemos obtener \cite{fundamentos_robo}:
    
    \textbf{Velocidad Angular }
    \begin{equation*}
        \boldsymbol{\omega}_i = \begin{cases}
            ^{i}\boldsymbol{R}_{i-1} (\boldsymbol{\omega}_{i-1} + \dot{q_i}\boldsymbol{z}_{i-1})$ , Si es de rotación$\\
            ^{i}\boldsymbol{R}_{i-1} (\boldsymbol{\omega}_{i-1})\  \  \  \  \ $  , Si es de prismático$
        \end{cases}
    \end{equation*}
    
    \textbf{Aceleración Angular }
    \begin{equation*}
        \dot{\boldsymbol{\omega}_i} = \begin{cases}
            ^{i}\boldsymbol{R}_{i-1} (\dot{\boldsymbol{\omega}}_{i-1} + \ddot{q_i}\boldsymbol{z}_{i-1} + \dot{q_i}\boldsymbol{w}_{i-1}\times\boldsymbol{z}_{i-1})$ , Si es de rotación$\\
            ^{i}\boldsymbol{R}_{i-1} (\dot{\boldsymbol{\omega}}_{i-1})\  \  \  \  \ $  , Si es de prismático$
        \end{cases}
    \end{equation*}
    
    \textbf{Aceleración Lineal}
    \begin{equation*}
        \ddot{\boldsymbol{p}}_i = ^{i}\boldsymbol{R}_{i-1}\ddot{\boldsymbol{p}}_{i-1} + \dot{\boldsymbol{\omega}}_i\times\boldsymbol{r}_{i-1,i}+\boldsymbol{\omega}_i\times(\boldsymbol{\omega}_i\times\boldsymbol{r}_{i-1,i})
    \end{equation*}   
    
    \textbf{Aceleración del centro de masa}
    \begin{equation*}
        \ddot{\boldsymbol{p}}_{c_i} = \ddot{\boldsymbol{p}}_{i} + \dot{\boldsymbol{\omega}}_i\times\boldsymbol{r}_{i,c_i}+\boldsymbol{\omega}_i\times(\boldsymbol{\omega}_i\times\boldsymbol{r}_{i,c_i})
    \end{equation*}   

    Para calcular estos valores adecuadamente se toma como valores iniciales\cite{barrientos_fundamentos_2007}: 

    $\boldsymbol{\omega}_0 = \begin{bmatrix}
        0\\0\\0
    \end{bmatrix}$ \ \ \ , $\dot{\boldsymbol{\omega}}_0 = \begin{bmatrix}
        0\\0\\0
    \end{bmatrix}$ \ \ \, $\dot{\boldsymbol{p}} =\begin{bmatrix}
        0\\0\\0
    \end{bmatrix} - \boldsymbol{g}$ \ \ \ , $\boldsymbol{z}_0 = \boldsymbol{z}_1 = ...= \boldsymbol{z}_n  = \begin{bmatrix}
        0\\0\\0
    \end{bmatrix}$
    
    El cálculo de la fuerza y de torques implicará un cálculo recursivo inverso, es decir, se tomará el valor actual para calcular el anterior.
    \begin{equation}
        \begin{cases}
            \boldsymbol{f}_i = ^{i}\boldsymbol{R}_{i+1} \boldsymbol{f}_{i+1} + m_i\boldsymbol{\ddot{p}}_{c_i}\\
            \boldsymbol{\mu}_i = ^{i}\boldsymbol{R}_{i+1} \boldsymbol{\mu}_{i+1} - \boldsymbol{f}_{i}\times(\boldsymbol{r}_{i-1,i}+\boldsymbol{r}_{i,c_i}) + ^{i}\boldsymbol{R}_{i+1}\boldsymbol{f}_{i+1}\times\boldsymbol{r}_{i,c_i} + \boldsymbol{I}_i\dot{\boldsymbol{\omega}}_i + {\boldsymbol{\omega}}_i\times\boldsymbol{I}_i\boldsymbol{\omega}_i
        \end{cases}
    \end{equation}

    De manera similar a las velocidades y aceleraciones, los valores iniciales estarán expresados por: 
    
    $\boldsymbol{\mu}_{N+1} = \begin{bmatrix}
        0\\0\\0
    \end{bmatrix}$ \ \ \ $\boldsymbol{f}_{N+1} = \begin{bmatrix} 
        0\\0\\0
    \end{bmatrix}$\ \ \ $^N\boldsymbol{R}_{N+1} = \begin{bmatrix}
        1&0&0\\0&1&0\\0&0&1
    \end{bmatrix}$

    Con las expresiones y términos calculados, podemos definir los torques generalizados del modelo dinámico y por consiguiente, el modelo dinámico:
    
    \begin{equation*}
        \boldsymbol{\tau}_i = {\boldsymbol{\mu}_i}^T {^{i}\boldsymbol{R}_{i-1}} 
        \boldsymbol{z}_{i-1}
    \end{equation*}
    
\section{Control por Modo Deslizante}


\subsection{Introducción al control por modo deslizante (SMC)}
El control por modo deslizante (SMC, por sus siglas en inglés) es un enfoque de control no lineal diseñado para dirigir el comportamiento de un sistema hacia un régimen dinámico específico, definido a través de una superficie de deslizamiento. El objetivo es que el sistema "deslice" sobre esta superficie, logrando así que la trayectoria del sistema se mantenga cerca de la deseada a pesar de perturbaciones y no linealidades. Aunque una de las desventajas de este método es el fenómeno de chattering, pequeñas oscilaciones indeseadas en la ley de control, el SMC es especialmente robusto y resistente a variaciones en el sistema y pequeñas perturbaciones. Esto lo convierte en una opción valiosa para aplicaciones en entornos inciertos o dinámicos, como es el caso de la cirugía robótica.

Este método tiene ventajas significativas para el control de sistemas robóticos en contextos donde existen incertidumbres y no linealidades, como en aplicaciones quirúrgicas para sutura. Su robustez permite mantener un rendimiento adecuado a pesar de variaciones inesperadas en el entorno, como el comportamiento dinámico de los tejidos. Además, el SMC es eficaz en el manejo de dinámicas no lineales, facilitando la adaptación a cambios sin necesidad de un modelo preciso del sistema. Su implementación es también relativamente sencilla, ya que se basa en el diseño de superficies de deslizamiento, reduciendo la complejidad matemática requerida en comparación con otros métodos. La flexibilidad en la definición de estas superficies permite adaptar el control a los requisitos específicos de cada tarea, optimizando así el desempeño de los manipuladores robóticos UR5 en entornos quirúrgicos \cite{utkin_sliding_mode_control}.  \cite{emergency_department_overload}

\subsection{Diseño del control SMC}

El \textbf{control por modo deslizante} (Sliding Mode Control, SMC) es una técnica robusta ampliamente utilizada para sistemas no lineales. Su objetivo principal es forzar al sistema a seguir una superficie de deslizamiento, donde la dinámica se vuelve robusta frente a perturbaciones y modelos inexactos. Para ello, se diseña una ley de control que garantice que la trayectoria del sistema converja hacia dicha superficie y permanezca en ella.

Primero se define una función de Lyapunov adecuada para garantizar estabilidad. En este caso, se emplea una función cuadrática:

\[
V(x) = \frac{1}{2} s^2
\]

donde la función de deslizamiento está dada por:

\[
s(x) = \dot{e} + \lambda e
\]

Aquí, \( e = x - x_d \) es el error entre el estado actual \( x \) y el deseado \( x_d \), y \( \lambda \) es una constante positiva que ajusta la convergencia.

Para asegurar estabilidad, deben cumplirse tres condiciones fundamentales:

\begin{itemize}
    \item \( V(x) > 0 \) para todo \( x \neq x_d \) (positividad).
    \item \( \dot{V}(x) < 0 \), lo que implica convergencia hacia la superficie de deslizamiento.
    \item El sistema debe alcanzar la superficie en tiempo finito y luego deslizarse sobre ella.
\end{itemize}

La derivada de Lyapunov es:

\[
\dot{V} = s \dot{s} = s(\ddot{e} + \lambda \dot{e})
\]

A partir del modelo dinámico del manipulador:

\[
\tau = M \cdot \ddot{q} + C \cdot \dot{q} + G
\]

se proyecta al espacio cartesiano mediante el Jacobiano \( J_a \), obteniendo:

\[
\ddot{x} = J_a \cdot M^{-1} \cdot (\tau - C \cdot \dot{q} - G) + \dot{J}_a \cdot \dot{q} \tag{1}
\]

La superficie de deslizamiento en el espacio cartesiano se define como:

\[
S = \dot{x} - \dot{x}_{\text{des}} + \lambda (x - x_{\text{des}})
\]

y su derivada:

\[
\dot{S} = \ddot{x} - \ddot{x}_{\text{des}} + \lambda (\dot{x} - \dot{x}_{\text{des}})
\]

La ley de control se diseña para que:

\[
\dot{S} = -k \cdot S - k_2 \cdot \tanh(k_3 \cdot S)
\]

con ganancias positivas \( k, k_2, k_3 \). Esto garantiza que \( \dot{V} = S^T \dot{S} < 0 \), cumpliendo la condición de estabilidad.

Sustituyendo la ecuación \((1)\) en la expresión de \( \dot{S} \), se obtiene la ecuación final para el torque de control:

\[
\tau = M \cdot J_a^{-1} \cdot \left[ -k \cdot S - k_2 \cdot \tanh(k_3 \cdot S) - \left( \dot{J}_a \cdot \dot{q} - \ddot{x}_{\text{des}} + \lambda (\dot{x} - \dot{x}_{\text{des}}) \right) \right] + C \cdot \dot{q} + G
\]

Esta ley de control garantiza el seguimiento deseado incluso en presencia de perturbaciones, aprovechando la robustez inherente del enfoque SMC.

\begin{itemize}
    \item \textbf{\( \tau \)}: Torque aplicado al sistema.
    \item \textbf{\( M \)}: Matriz de masa (o inercia).
    \item \textbf{\( \ddot{q} \)}: Aceleración generalizada.
    \item \textbf{\( \dot{q} \)}: Velocidad generalizada.
    \item \textbf{\( q \)}: Posición articular.
    \item \textbf{\( C \)}: Matriz de Coriolis y centrífuga.
    \item \textbf{\( G \)}: Vector de fuerzas de gravedad.
    \item \textbf{\( J_a \)}: Jacobiano de la configuración.
    \item \textbf{\( \dot{J}_a \)}: Derivada temporal del Jacobiano.
    \item \textbf{\( x \)}: Posición en el espacio cartesiano.
    \item \textbf{\( \dot{x} \)}: Velocidad cartesiana.
    \item \textbf{\( \ddot{x} \)}: Aceleración cartesiana.
    \item \textbf{\( x_{\text{des}} \)}, \( \dot{x}_{\text{des}} \), \( \ddot{x}_{\text{des}} \): Referencias deseadas.
    \item \textbf{\( \lambda \)}: Ganancia de la superficie de deslizamiento.
    \item \textbf{\( S \)}: Función de deslizamiento.
    \item \textbf{\( \dot{S} \)}: Derivada de la función de deslizamiento.
    \item \textbf{\( k \)}, \textbf{\( k_2 \)}, \textbf{\( k_3 \)}: Ganancias del controlador.
    \item \textbf{\( \tanh \)}: Función tangente hiperbólica, usada para suavizar el control discontinuo.
    \item \textbf{\( \# \)}: Símbolo que representa la pseudoinversa (si aplica).
\end{itemize}



\section{Control por Impedancia}

\begin{equation}
    \boldsymbol{\tau} = \boldsymbol{M} \boldsymbol{J}^{-1} \left[ \left( \ddot{\boldsymbol{x}}_d - \boldsymbol{K}_p \boldsymbol{e}_x - \boldsymbol{K}_d \boldsymbol{e}_{\dot{x}} \right) - \boldsymbol{J} \dot{\boldsymbol{q}} \right] + \boldsymbol{h}(\boldsymbol{q}, \dot{\boldsymbol{q}})
\end{equation}

$Donde:$

\begin{align*}
\bm{\tau} &\in \mathbb{R}^{6} \quad \text{: Vector de torques articulares} \\
\bm{M} &\in \mathbb{R}^{6 \times 6} \quad \text{: Matriz de inercia articular} \\
\bm{J}^{\dagger} &\in \mathbb{R}^{6 \times 7} \quad \text{: Pseudo-inversa del Jacobiano extendido} \\
\bm{J} &\in \mathbb{R}^{7 \times 6} \quad \text{: Jacobiano extendido (posición + cuaternión)} \\
\ddot{\bm{x}}_d &\in \mathbb{R}^{7} \quad \text{: Aceleración cartesiana deseada} \\
\bm{K}_p &\in \mathbb{R}^{7 \times 7} \quad \text{: Matriz de ganancia proporcional} \\
\bm{K}_d &\in \mathbb{R}^{7 \times 7} \quad \text{: Matriz de ganancia derivativa} \\
\bm{e}_x &\in \mathbb{R}^{7} \quad \text{: Error de pose (posición + orientación)} \\
\bm{e}_{\dot{x}} &\in \mathbb{R}^{7} \quad \text{: Error de velocidad cartesiana} \\
\bm{h}(\bm{q}, \dot{\bm{q}}) &\in \mathbb{R}^{6} \quad \text{: Términos no lineales (Coriolis + gravedad)}
\end{align*}
\input{secciones/capitulo3}
\chapter{MARCO METODOLÓGICO}

En este capítulo se describe de manera detallada la metodología empleada para diseñar, implementar y evaluar un sistema robótico teleoperado con retroalimentación háptica, orientado a mejorar la precisión y seguridad en procedimientos de sutura en heridas superficiales. La ruta metodológica sigue un enfoque experimental, en el que se desarrollan y validan componentes mediante simulaciones en entornos que toman en consideración variables físicas como la masa, gravedad e inercia y análisis cuantitativos que aseguren la viabilidad técnica del sistema acorde a los parámetros de desempeño del controlador para el seguimiento de la trayectoria.


\section{Metodología de desarrollo basada en la norma VDI 2206}

El desarrollo del sistema robótico se estructura siguiendo la norma VDI 2206, una metodología reconocida para el diseño de sistemas mecatrónicos que propone un enfoque en “V” modificado. Esta metodología permite una integración temprana y concurrente de los dominios mecánico, electrónico y de control, promoviendo la validación continua desde las etapas iniciales del diseño.

La metodología contempla una fase exploratoria, donde se analizan los principios físicos, conceptos funcionales y alternativas tecnológicas, y una fase sistemática de diseño, donde se desarrollan modelos del sistema, simulaciones, implementación físic a y validación final.

En este proyecto, la fase exploratoria incluyó el análisis de los requerimientos del procedimiento de sutura, la evaluación de dispositivos candidatos (UR5, Geomagic Touch) y la definición preliminar de principios físicos relevantes (fuerzas de interacción, precisión posicional, escalamiento de movimientos). Posteriormente, en la fase de desarrollo sistemático, se aplicó la metodología en cuatro etapas clave:

 \begin{enumerate}
    \item Diseño conceptual: definición de arquitectura del sistema, identificación de comunicación entre componentes y especificaciones funcionales.
    \item Modelado y simulación: desarrollo del modelo dinámico del UR5 y simulación del entorno en Gazebo con los elementos 3d añadidos.
    \item Implementación: integración de nodos ROS 2, C++, librerías oficiales de los equipos utilizados y protocolos de comunicación entre estos.
    \item Validación: pruebas en escenarios reales en el laboratorio de Ingeniería Mecatrónica para medir precisión, respuesta dinámica y calidad de la retroalimentación háptica.
\end{enumerate}

\begin{figure}[H]
    \centering
    \includegraphics[width=1\linewidth]{images/vdi_muy_basico.png}
    \caption{Esquema de la metodología VDI 2206}
    \label{fig:enter-label}
\end{figure}

\section{Diseño del sistema robótico}

En esta fase inicial se establecen los requerimientos funcionales y técnicos del sistema robótico, considerando las demandas propias del procedimiento de sutura. Se seleccionan los dispositivos principales, incluyendo el manipulador UR5 y el dispositivo háptico Geomagic Touch, definiendo sus roles dentro del sistema. Además, se desarrolla un modelo cinemático y dinámico del UR5 utilizando el método de Denavit-Hartenberg, lo que permite determinar la posición en tiempo real del efector final con base en los valores articulares y facilitar su integración en la plataforma de simulación.

Paralelamente, se configura un entorno virtual en Gazebo v11, utilizando ROS 2 Humble como marco de trabajo. Este entorno replica las condiciones físicas reales, como la gravedad, las colisiones y las interacciones con objetos simulados, permitiendo realizar pruebas preliminares en un contexto controlado y seguro. La simulación incluye elementos como bloques, mesas, sillas y diversas herramientas, extraídas de bibliotecas públicas, para representar de manera realista las tareas que llevará a cabo el sistema.

\begin{figure}[H]
    \centering
    \includegraphics[width=1\linewidth]{images/comunication_desired.png}
    \caption{Comunicación de tópicos y nodos planteada}
    \label{fig:enter-label}
\end{figure}

\section{Implementación del sistema de control}

El desarrollo del control se centra en dos dispositivos clave: el manipulador UR5 y el Geomagic Touch. Para el UR5, se implementa un controlador basado en dinámica inversa que permita generar movimientos precisos en función de las entradas de posición y velocidad admitidas por el fabricante. Este controlador se complementa con un módulo de optimización para ajustar los parámetros de desempeño, logrando un balance adecuado entre velocidad de respuesta y precisión.

En el caso del dispositivo háptico, se calibra su capacidad para enviar y recibir señales de posición y fuerza, estableciendo un puente efectivo entre el usuario y el manipulador robótico. Además, se desarrollan transformaciones que escalan los movimientos del dispositivo háptico al espacio de trabajo del UR5, garantizando que las acciones del operador se reflejen fielmente en la teleoperación.

\subsection{Control con optimización}

Para poder alcanzar la posición deseada del efector final del manipulador UR5 se plantea un problema de minimización cuadrática ponderado para dos tareas, minimizar el error de posición y el error de orientación respecto a la velocidad articular. Se definen dos matrices de pesos $\boldsymbol{W_p}$ para la posición y $\boldsymbol{W_o}$ para la orientación. 
    \begin{mini*}|s|[0]
        {\boldsymbol{\dot{q}}}{\boldsymbol{W_p}\|\boldsymbol{J_p(q)}\boldsymbol{\dot{q}} - \boldsymbol{e_p}\|^2 + \boldsymbol{W_o}\|\boldsymbol{J_o(q)}\boldsymbol{\dot{q}} - \boldsymbol{e_o}\|^2}
        {}
        {\label{eq:minimizationProblem1}}
    \end{mini*}
    
    Se define la \textbf{matriz Hessiana}:
    
    $\boldsymbol{H} =\boldsymbol{W_p}\boldsymbol{J_p(q)}^{T}\boldsymbol{W_p}\boldsymbol{J_p(q)} + \boldsymbol{W_o}\boldsymbol{J_o(q)}^{T}\boldsymbol{W_o}\boldsymbol{J_o(q)} $,  

    donde $\boldsymbol{J_p(q)}$ es la parte del Jacobiano asociada a la posición del efector final y $\boldsymbol{J_o(q)}$ es la parte del Jacobiano asociada a su orientación  
    
    y el \textbf{gradiente} 
    
    $\nabla g(\boldsymbol{q}) = -\boldsymbol{W_p}\boldsymbol{J_p(q)}^{T}\boldsymbol{W_p}\boldsymbol{e_p}-\boldsymbol{W_o}\boldsymbol{J_o(q)}^{T}\boldsymbol{W_o}\boldsymbol{e_o}$ 

    donde $\boldsymbol{e_p}$ es el error de la posición del efector final y $\boldsymbol{e_o}$ es su error de orientación(representado en parte vectorial).
    
    Siendo la función a minimizar 
    \begin{mini}|s|[0]
        {\boldsymbol{\dot{q}}}{\frac{1}{2}\boldsymbol{\dot{q}}^{T}\boldsymbol{H} + \nabla g(\boldsymbol{q})\boldsymbol{\dot{q}}}
        {}
        {\label{eq:minimizationProblem}}
    \end{mini}

    Finalmente, se puede aplicar recursividad mediante un parámetro $\alpha$ que determinará el tamaño de paso en el avance del optimizador hasta encontrar los valores articulares adecuados.
    \begin{equation}
        \boldsymbol{q}_{k+1} = \boldsymbol{q}_k + \alpha\boldsymbol{\dot{q}}_k
    \end{equation}
    

\section{Validación experimental}

La fase final está dedicada a la evaluación del sistema en simulaciones realistas. Se diseñan experimentos que reproducen escenarios de sutura, variando las propiedades mecánicas de los bloques simulados para medir el desempeño del sistema bajo diferentes condiciones. Durante las pruebas, se registran métricas clave como error posicional, tiempo de respuesta y calidad de la retroalimentación háptica.

Los resultados obtenidos como el tiempo de sutura, la fiabilidad de la retroalimentación háptica y la calidad de la sutura se analizan para determinar la precisión y confiabilidad del sistema propuesto . Asimismo, se identifican posibles limitaciones técnicas que podrían abordarse en futuras iteraciones.
\chapter{RESULTADOS}

En este capítulo se mostrarán los resultados de las pruebas realizadas sobre el banco de trabajo para el seguimiento de trayectorias en un entorno controlado. Asimismo, se compara el comportamiento de las arquitecturas de control propuestas en trayectorias curvas y


los controladores usados para el control cinemático y el método utilizado para la comunicación del dispositivo háptico con el entorno simulado.



\section{Simulación del brazo robótico UR5 en Gazebo}
Para llevar a cabo las pruebas de movilidad y seguimiento a la trayectoria, se llevaron a cabo pruebas en el entorno Gazebo utilizando el sistema ROS 2 Humble, el cual posibilita leer y enviar acciones a parámetros como rotación o velocidad, teniendo en cuenta características físicas como masa o inercia del robot y simulando los efectos de fricción o colisión.

Se logró cargar el modelo del robot UR5, sub modelo e, y asegurar el correcto cargado de los paquetes necesarios para la simulación. Así como los programas que contienen los comandos para enviar movimiento a las articulaciones.

\begin{figure}[H]
    \centering
    \includegraphics[width=0.75\linewidth]{images/gaz_ur5e.png}
    \caption{Simulación en Gazebo}
    \label{fig:enter-label}
\end{figure}


\section{Diseño de controlador por cinemática inversa y diferencial}

Se modeló un sistema de control cinemático, puesto que el robot ur5 solo admite la entrada de señales de posición o velocidad. El sistema de control hace uso de cinemática directa para convertir las posiciones angulares de las articulaciones al espacio cartesiano y poder calcular el error y posteriormente el cambio del error. Esto para poder caracterizar al controlador con un control proporcional y derivativo, permitiendo el seguimiento a una trayectoria deseada.

Se logró la implementación del controlador en el entorno simulado de Gazebo obteniendo un error al seguimiento en los 3 ejes cartesianos dentro de un rango de 0.01 m con una referencia sinusoidal

\begin{figure}[H]
    \centering
    \includegraphics[width=0.75\linewidth]{images/Pos.png}
    \caption{Seguimiento a la trayectoria en los 3 ejes}
    \label{fig:enter-label}
\end{figure}

\begin{figure}[H]
    \centering
    \includegraphics[width=0.75\linewidth]{images/errores.png}
    \caption{Error al seguimiento a la trayectoria}
    \label{fig:enter-label}
\end{figure}

\section{Integración del dispositivo háptico para la teleoperación del efector final}

Dado que se requiere una alta frecuencia para la comunicación de datos entre los dispositivos, se utilizaron librerías principalmente basadas en C/C++. De esta forma, se logró mantener una comunicación con 1 kHz de frecuencia para la lectura de la posición y orientación del efector final, los ángulos articulares de cada una de las articulaciones que dispone, y el comando para requerir las fuerzas de los 3 grados de libertad actuados que dispone. Esta librería cuenta con las dependencias de las funciones creadas por la empresa 3D Systems para la manipulación a bajo nivel de las características del dispositivo háptico. De esta forma, se obtuvieron los nodos y tópicos mostrados en la Figura \ref{fig:haptic_topics}.

Por otro lado, dado que la lectura de la posición en su estado base (Figura \ref{fig:Geomagic_touch_base}) del efector final se encuentra definida como el origen en su sistema de referencia, no será posible comunicar directamente las posiciones para la manipulación del dispositivo háptico. De esta forma, se calibró una posición inicial deseada por el usuario como un punto de origen para tomar en consideración únicamente la diferencia de posición actual y el origen del módulo háptico.

\begin{figure}[H]
    \centering
    \includegraphics[width=1\linewidth]{images/haptic_topics.jpeg}
    \caption{Diagrama de nodos y tópicos para el control del Geomagic Touch}
    \label{fig:haptic_topics}
\end{figure}

Se calibraron los parámetros de desempeño del controlador que brinda el UR5 por defecto de fábrica para que establezca la posición deseada en 0.1 segundos y comande la posición cada 0.5 segundos. Estos parámetros se determinaron para que el seguimiento sea lo más rápido posible sin perder precisión, dando suficiente tiempo para que se pueda establecer en la posición deseada. Además, cabe resaltar que en estas primeras pruebas solo se consideró el control de la posición, sin la orientación.


\begin{figure}[H]
    \centering
    \includegraphics[width=0.7\linewidth]{images/teleop_square.png}
    \caption{Seguimiento de trayectoria cuadrada con el efector final del dispositivo háptico}
    \label{fig:teleop_square}
\end{figure}

Como se puede observar en la Figura \ref{fig:teleop_square}, se logró el seguimiento de la trayectoria al intentar realizar un cuadrado con el efector final del dispositivo háptico. Asimismo, en las gráficas mostradas en las Figuras \ref{fig:x_square_teleop}, \ref{fig:y_square_teleop} y \ref{fig:z_square_teleop}, se puede observar que el error durante el seguimiento de trayectoria resultó en todo momento menor a 10 cm. Dado el carácter de la investigación, se buscará que el controlador sea más preciso para que el cirujano pueda mantener un error mínimo durante todo el seguimiento de la trayectoria. Sin embargo, se observó que el desfase para el seguimiento de la trayectoria resultó menor a 0.5 segundos, lo cual resulta óptimo para el seguimiento.

Se considera que, aunque estos parámetros fueron simulados en un entorno de simulación realista, se tendrán que calibrar en el dispositivo real, dado que este puede contar con ligeras variaciones mecánicas que podrían afectar el desempeño del controlador propuesto.

\begin{figure}[H]
    \centering
    \includegraphics[width=0.6\linewidth]{images/x_haptic_teleop.png}
    \caption{Seguimiento en el eje X de la trayectoria cuadrada}
    \label{fig:x_square_teleop}
\end{figure}

\begin{figure}[H]
    \centering
    \includegraphics[width=0.6\linewidth]{images/y_haptic_teleop.png}
    \caption{Seguimiento en el eje Y de la trayectoria cuadrada}
    \label{fig:y_square_teleop}
\end{figure}

\begin{figure}[H]
    \centering
    \includegraphics[width=0.6\linewidth]{images/z_haptic_teleop.png}
    \caption{Seguimiento en el eje Z de la trayectoria cuadrada}
    \label{fig:z_square_teleop}
\end{figure}

\section{Integración de teleoperación y control por optimización en entorno físico}

El sistema completo consta de la comunicación por medio de ROS2 de los nodos del módulo Geomagic Touch y los nodos de control del UR5; mediante el control por optimización HQP se implementó el seguimiento a la referencia de posición en el espacio cartesiano del módulo háptico transformado al espacio de trabajo del robot UR5 y tomando como posición inicial la del efector final del robot.

\begin{figure}[H]
    \centering
    \includegraphics[width=.75\linewidth]{images/implementacionXYZ.png}
    \caption{Seguimiento en el espacio cartesiano}
    \label{fig:xyz_impl}
\end{figure}

\begin{figure}[H]
    \centering
    \includegraphics[width=0.5\linewidth]{images/impl_X.png}
    \caption{Error en X [mm] vs Tiempo [s]}
    \label{fig:err_x}
\end{figure}

\begin{figure}[H]
    \centering
    \includegraphics[width=0.5\linewidth]{images/err_y_impl.png}
    \caption{Error en Y [mm] vs Tiempo [s]}
    \label{fig:enter-label}
\end{figure}

\begin{figure}[H]
    \centering
    \includegraphics[width=0.5\linewidth]{images/err_z_imp.png}
    \caption{Error en Z [mm] vs Tiempo [s]}
    \label{fig:enter-label}
\end{figure}



\section{Diseño y fabricación del adaptador de una pinza genérica para el gripper Robotiq 2F‑85 para UR5}

El diseño de las piezas se realizó utilizando el software Fusion 360 de la empresa Autodesk. Asímismo, se utilizó instrumentos de presición de 0.01 militetros para la medición de los elementos físcos que se buscan adaptar. Estas piezas 3D fueron impresas con PLA al 50\% de desidad en el laboratorio de mecatrónica 201 de la UTEC.

Las Figuras \ref{fig:vistas_soporte_pinza}, \ref{fig:vistas_pinza_slider}, \ref{fig:vistas_pinza_seguro_inf} y \ref{fig:vistas_pinza_seguro_sup} muestran las vistas principales del soporte diseñado para la adaptación mecánica de una pinza genérica.

\begin{figure}[H]
    \centering

    \begin{minipage}{0.32\textwidth}
        \centering
        \includegraphics[width=\linewidth]{secciones/pinza_gripper_frontal.png}
        \caption*{(a) Vista frontal}
    \end{minipage}
    \hfill
    \begin{minipage}{0.32\textwidth}
        \centering
        \includegraphics[width=\linewidth]{secciones/pinza_gripper_lateral.png}
        \caption*{(b) Vista lateral}
    \end{minipage}
    \hfill
    \begin{minipage}{0.32\textwidth}
        \centering
        \includegraphics[width=\linewidth]{secciones/pinza_gripper_superior.png}
        \caption*{(c) Vista superior}
    \end{minipage}

    \caption{Vistas principales del diseño del soporte principal de la pinza.}
    \label{fig:vistas_soporte_pinza}
\end{figure}

\begin{figure}[H]
    \centering

    \begin{minipage}{0.32\textwidth}
        \centering
        \includegraphics[width=\linewidth]{images/dezlizador_frontal.png}
        \caption*{(a) Vista frontal}
    \end{minipage}
    \hfill
    \begin{minipage}{0.32\textwidth}
        \centering
        \includegraphics[width=\linewidth]{images/dezlizador_lateral.png}
        \caption*{(b) Vista lateral}
    \end{minipage}
    \hfill
    \begin{minipage}{0.32\textwidth}
        \centering
        \includegraphics[width=\linewidth]{images/dezlizador_superior.png}
        \caption*{(c) Vista superior}
    \end{minipage}

    \caption{Vistas principales del diseño del slider de la pinza.}
    \label{fig:vistas_pinza_slider}
\end{figure}

\begin{figure}[H]
    \centering

    \begin{minipage}{0.32\textwidth}
        \centering
        \includegraphics[width=\linewidth]{images/seguro_pinza_inf_frontal.png}
        \caption*{(a) Vista frontal}
    \end{minipage}
    \hfill
    \begin{minipage}{0.32\textwidth}
        \centering
        \includegraphics[width=\linewidth]{images/seguro_pinza_inf_lateral.png}
        \caption*{(b) Vista lateral}
    \end{minipage}
    \hfill
    \begin{minipage}{0.32\textwidth}
        \centering
        \includegraphics[width=\linewidth]{images/seguro_pinza_inf_superior.png}
        \caption*{(c) Vista superior}
    \end{minipage}

    \caption{Vistas principales del diseño del seguro inferior que acopla la pinza.}
    \label{fig:vistas_pinza_seguro_inf}
\end{figure}


\begin{figure}[H]
    \centering

    \begin{minipage}{0.32\textwidth}
        \centering
        \includegraphics[width=\linewidth]{images/seguro_pinza_sup_frontal.png}
        \caption*{(a) Vista frontal}
    \end{minipage}
    \hfill
    \begin{minipage}{0.32\textwidth}
        \centering
        \includegraphics[width=\linewidth]{images/seguro_pinza_sup_lateral.png}
        \caption*{(b) Vista lateral}
    \end{minipage}
    \hfill
    \begin{minipage}{0.32\textwidth}
        \centering
        \includegraphics[width=\linewidth]{images/seguro_pinza_sup_superior.png}
        \caption*{(c) Vista superior}
    \end{minipage}

    \caption{Vistas principales del diseño del seguro superior que acopla la pinza.}
    \label{fig:vistas_pinza_seguro_sup}
\end{figure}

La Figura \ref{fig:vistas_pinza_ensamble} muestra el resultado final del acople mecánico sobre el gripper 2F85.

\begin{figure}[H]
    \centering

    \begin{minipage}{0.9\textwidth}
        \centering
        \includegraphics[width=\linewidth]{images/ensamble_try3 v11.png}
        \caption*{(a)  Ensamblaje final gripper 2f 85 de la empresa Robotiq}
    \end{minipage}
    \hfill
    \begin{minipage}{0.9\textwidth}
        \centering
        \includegraphics[width=\linewidth]{images/handegripp v6.png}
        \caption*{(b) Ensamblaje final gripper hand-e de la empresa Robotiq}
    \end{minipage}
   \caption{Vistas principales del ensamble final.}
    \label{fig:vistas_pinza_ensamble}
\end{figure}

\section{Diseño y fabricación del adaptador de agarre para el efector final del Geomagic Touch}

Las Figuras \ref{fig:vistas_haptic1}, \ref{fig:vistas_haptic2} y \ref{fig:vistas_haptic3} muestran las vistas principales del agarre diseñado para el efector final del dispositivo háptico.


\begin{figure}[H]
    \centering

    \begin{minipage}{0.32\textwidth}
        \centering
        \includegraphics[width=\linewidth]{images/pinza_anular_frontal.png}
        \caption*{(a) Vista frontal}
    \end{minipage}
    \hfill
    \begin{minipage}{0.32\textwidth}
        \centering
        \includegraphics[width=\linewidth]{images/pinza_anular_lateral.png}
        \caption*{(b) Vista lateral}
    \end{minipage}
    \hfill
    \begin{minipage}{0.32\textwidth}
        \centering
        \includegraphics[width=\linewidth]{images/pinza_anular_superior.png}
        \caption*{(c) Vista superior}
    \end{minipage}

    \caption{Vistas principales del diseño del agarre del dedo anular.}
    \label{fig:vistas_haptic1}
\end{figure}

\begin{figure}[H]
    \centering

    \begin{minipage}{0.32\textwidth}
        \centering
        \includegraphics[width=\linewidth]{images/pinza_pulgar_frontal.png}
        \caption*{(a) Vista frontal}
    \end{minipage}
    \hfill
    \begin{minipage}{0.32\textwidth}
        \centering
        \includegraphics[width=\linewidth]{images/pinza_pulgar_lateral.png}
        \caption*{(b) Vista lateral}
    \end{minipage}
    \hfill
    \begin{minipage}{0.32\textwidth}
        \centering
        \includegraphics[width=\linewidth]{images/pinza_pulgar_superior.png}
        \caption*{(c) Vista superior}
    \end{minipage}

    \caption{Vistas principales del diseño del agarre del dedo pulgar.}
    \label{fig:vistas_haptic2}
\end{figure}


\begin{figure}[H]
    \centering

    \begin{minipage}{0.32\textwidth}
        \centering
        \includegraphics[width=\linewidth]{images/tapa_cilindro_frontal.png}
        \caption*{(a) Vista frontal}
    \end{minipage}
    \hfill
    \begin{minipage}{0.32\textwidth}
        \centering
        \includegraphics[width=\linewidth]{images/tapa_cilindro_lateral.png}
        \caption*{(b) Vista lateral}
    \end{minipage}
    \hfill
    \begin{minipage}{0.32\textwidth}
        \centering
        \includegraphics[width=\linewidth]{images/tapa_cilindro_superior.png}
        \caption*{(c) Vista superior}
    \end{minipage}

    \caption{Vistas principales del diseño de la tapa del acople.}
    \label{fig:vistas_haptic3}
\end{figure}

La Figura \ref{fig:vistas_pinza123} muestra el ensamblaje final de las piezas para formar el acople para el dispositivo háptico.

\begin{figure}[H]
    \centering

    \begin{minipage}{0.49\textwidth}
        \centering
        \includegraphics[width=\linewidth]{images/haptic_explo.png}
        \caption*{(a) Plano de explosión}
    \end{minipage}
    \hfill
    \begin{minipage}{0.49\textwidth}
        \centering
        \includegraphics[width=\linewidth]{images/haptic_ensamble.png}
        \caption*{(b) Ensamblaje final}
    \end{minipage}
   \caption{Vistas principales del ensamblaje final.}
    \label{fig:vistas_pinza123}
\end{figure}

\section{Pruebas con el banco de trabajo}

Se realizaron pruebas de seguimiento a la trayectoria con las 3 arquitecturas de control para medir los errores máximos y promedios es el espacio de trabajo
\customchapter{CONCLUSIONES} 
El Banco de pruebas implementado para realizar las pruebas de sutura integran un sistema de Teleoperación para el robot UR5 que permite comandar al robot con Posiciones y Orientaciones distintas a lo largo de una trayectoria creada desde la herramienta Háptica Geomagic Touch. Para lograr un correcto funcionamiento de este sistema se realiza el diseño de 3 arquitecturas de control basadas en un control cinemático y dos controles Dinámicos. El control cinemático usa un optimizador Jerárquico y los controles dinámicos se modelan con base en el Controlador por Modos Deslizantes y el control por Impedancia. 

Durante las pruebas con el control cinemático, se observó un error máximo de ±0.05 metros en los tres ejes cartesianos durante trayectorias con curvas cerradas o movimientos bruscos. A pesar de estos errores, el tiempo de procesamiento y cálculo de las posiciones articulares fue notablemente bajo, menor a 8 ms en trayectorias suaves. Esto permitió una frecuencia de envío de datos al robot de 0.1 kHz, lo cual es esencial para un comportamiento preciso en tareas que demandan alta exactitud.

La implementación del controlador basado en optimización Jerárquica (HQP) demostró una mejora significativa en la precisión. Los errores en los tres ejes cartesianos del espacio de trabajo se redujeron a un máximo de 8 mm, manteniéndose dentro del rango de precisión establecido para un seguimiento de trayectoria adecuado.

En contraste, las arquitecturas de control dinámicas, específicamente el control dinámico SMC y el control por impedancia, mostraron errores máximos de 18 mm y 40 mm, respectivamente, en el seguimiento de trayectorias. Si bien estos valores se encuentran dentro del rango aceptable, son considerablemente superiores a los obtenidos con el control cinemático con optimizador, lo que sugiere que para tareas que priorizan la precisión en el seguimiento, las arquitecturas dinámicas podrían requerir una sintonización más fina o ser menos adecuadas.

Finalmente, las pruebas de suturas realizadas sobre un kit de cirugía laparoscópica, integrando todos los elementos del banco de prueba, confirmaron la viabilidad del sistema en un entorno controlado. Se logró un tiempo medio de finalización de 5 minutos con errores máximos de 8 mm en el seguimiento de trayectoria. Es importante destacar la aparición de 3 episodios de Jacobianos indeterminados durante movimientos bruscos, un aspecto a considerar para la robustez del sistema en situaciones dinámicas.

\input{secciones/recomendaciones} % sección opcional

%% ============================================================================
\renewcommand{\bibname}{\hfill\Large\bf{REFERENCIAS BIBLIOGRÁFICAS}\hfill}

\bibliographystyle{IEEEtran} % Estableciendo el estilo de citas IEEEtram.

\bibliography{referencias} % Recibe las referencias de IEEE

\chapter*{\center \Large ANEXOS} 
\addcontentsline{toc}{section}{\bfseries ANEXOS} 
\markboth{ANEXOS}{ANEXOS} 

\par Los algoritmos desarrollados .....
    

\end{document}
